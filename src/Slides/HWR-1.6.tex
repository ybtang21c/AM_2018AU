% !TEX root = ../1-main-SL.tex
% !TEX encoding = UTF-8  (utf8)

\begin{frame}
	\centering
	\bf\Huge\color{purple} 习题讲评\\[1em]
	\small 习题1-6\\[1cm]
	\small\color{gray}2018-11-9
\end{frame}

\section{说在前面}

\begin{frame}[t]\frametitle{说在前面}
	\linespread{1.8}
	\Large
	\vspace*{-1em}
    \begin{itemize}
    	\item 几个小问题
    	\begin{enumerate}
    		\item {\it 暂时不要用L'Hospital法则}
    		\item {\color{purple}$n\to\infty$时,$\sqrt{n+i}=\sqrt{n}$}
    		\item {\color{purple} 错误的放缩:
    			$$\sum\limits_{i=1}^kA_i\sqrt{n+1}
    			\leq\sum\limits_{i=1}^kA_i\sqrt{n+i}
    			\leq\sum\limits_{i=1}^kA_i\sqrt{n+k}$$
    		}
    	\end{enumerate}
    	\item \baa{作业不订正或者订正不认真一律C}
    \end{itemize}
\end{frame}

\section{参考解答}

\subsection{习题1-6}

\begin{frame}[t]\frametitle{1.计算极限}
\large
(1)$\limx{\pi}\df{\sin 7x}{\cos 4x}
=\df{\limx{\pi}\sin 7x}
	{\limx{\pi}\cos 4x}=\df{\sin7\pi}{\cos4\pi}=\df01=0$

\bs
(2)$\limx0\df{(1+x)^{1+x}+x\sin\df1x}{e^x+\sin x}
=\df{\limx0(1+x)^{1+x}+\limx0x\sin\df1x}
{\limx0e^x+\limx0\sin x}=1$

\bs
(3)$\limx{a}\df{\cos x-\cos a}{x-a}
=-\limx a\sin\df{x+a}2
\limx a\df{\sin\frac{x-a}2}{\frac{x-a}2}=-\sin a$

\bs
(4)$\limx0\df{1-\cos3x}{x\sin x}
=\limx0\df{2\sin^2\frac32x}
{\left(\frac{3}{2}x\right)^2}\df{\left(\frac32x\right)^2}{x^2}
\df{x}{\sin x}=\df92$
\end{frame}

\begin{frame}[t]\frametitle{1.计算极限}
\large
(5)
$
\limx1\df{\sqrt[3]x-1}{\sqrt x-1}
\xlongequal{y=x-1}\lim\limits_{y\to0}\df{y}{\sqrt{y+4}-2}$\\
$\hspace*{3cm}=\lim\limits_{y\to0}\df{y(\sqrt{y+4}+2)}{y}=4
$

\bs
(6)$\limx1\df{\sqrt[3]x-1}{\sqrt x-1}
=\limx1\df{(\sqrt{x}+1)(x-1)}
{(x-1)(x^{\frac23}+x^{\frac13}+1)}=\df23$

\bs
(7)$\limn\left(1+\df1n\right)^{n-6}
=\limn\left(1+\df1n\right)^n
\left[\limn\left(1+\df1n\right)\right]^{-6}=e$
\end{frame}

\begin{frame}[t]\frametitle{1.计算极限}
\large
(8)$\limn\left(\df{n}{n^2+1}+\df{n}{n^2+2}
+\ldots+\df{n}{n^2+n}\right)$

解:{\it 注意到
$$\df{n^2}{n^2+n}<\df{n}{n^2+1}+\df{n}{n^2+2}
+\ldots+\df{n}{n^2+n}<\df{n^2}{n^2+1},$$
而$\limn\df{n^2}{n^2+n}=\df{n^2}{n^2+1}=1$,故由夹逼准则
$\mbox{原式}=1$}

\bs
(9)$\limx0(1-x)^{\frac1{\sin x}}
=\limx0\left[(1-x)^{-\frac1x}\right]^{\frac{-x}{\sin x}}$\\
$\hspace*{3cm}=\left\{\limx0\left[(1-x)^{-\frac1x}\right]\right\}
^{\limx0\frac{-x}{\sin x}}=e^{-1}$
\end{frame}

\begin{frame}[t]\frametitle{1.计算极限}
\large
(10)$\limx{\infty}\left(\df{1+x}{2+x}\right)^{2x}
=\df{\left[\limx{\infty}\left(\df1x+1\right)^x\right]^2}
{\left[\limx{\infty}\left(\df2x+1\right)^{\frac x2}\right]^4}
=\df{e^2}{e^4}=e^{-2}$

\bs
(11)$\limx{+\infty}\left(\df{1+x}{1+2x}\right)^{2x}
=\limx{+\infty}\left(\df12\right)^{2x}
\df{\left[\limx{+\infty}\left(\df1x+1\right)^x\right]^2}
{\limx{+\infty}\left(\df1{2x}+1\right)^{2x}}$\\
$\hspace*{3cm}=0\cdot\df{e^2}e=0.$\fin
\end{frame}

\begin{frame}[t]\frametitle{2.递推数列的极限}
\large
2.证明数列
$$a_n=\underbrace{\sqrt{2+\sqrt{2+\ldots+\sqrt2}}}
_{n\mbox{\footnotesize 个}2}$$
收敛,并求其极限。

\bs
解:\it 显然$\{a_n\}$严格单调递增。又
$$a_n=\underbrace{\sqrt{2+\sqrt{2+\ldots+\sqrt2}}}
_{n\mbox{\footnotesize 个}2}
<\underbrace{\sqrt{2+\sqrt{2+\ldots+\sqrt4}}}
_{n-1\mbox{\footnotesize 个}2}=2,$$
故由单调有界准则,可知$\{a_n\}$必收敛。设其极限为$A$。
\end{frame}

\begin{frame}[t]\frametitle{2.递推数列的极限}
\linespread{1.6}
\large
\it 
注意到有递推式
$$a_{n+1}=\sqrt{2+a_n},$$
在其两边同时令$n\to\infty$,可得
$$A=\sqrt{2+A},$$
进而可解得$A=2$或$A=-1$。因为$a_n$均非负,故由极限的保号性
$A\geq 0$,从而$A=-1$不可能是所求极限。

综上,$\limn a_n=2$.\fin
\end{frame}

\begin{frame}[t]\frametitle{3.极限的运算}
\large

3.已知$A_1+A_2+\ldots+A_k=0$,证明
$$\limn (A_1\sqrt{n+1}+A_2\sqrt{n+2}+\ldots+A_k\sqrt{n+k})=0.$$

解:\it 对任意$l=2,3,\ldots,k$,
$$\limn(\sqrt{n+l}-\sqrt{n+1})
=\limn\df{l-1}{\sqrt{n+l}+\sqrt{n+1}}=0.$$
进而\small
\begin{align*}
	\mbox{原式}
	&=\limn\left[A_2\left(\sqrt{n+2}-\sqrt{n+1}\right)
	+A_3\left(\sqrt{n+3}-\sqrt{n+1}\right)\right.\\
	&\left.\quad+\ldots
	+A_k\left(\sqrt{n+k}-\sqrt{n+1}\right)\right]\\
	&=A_2\limn\left(\sqrt{n+2}-\sqrt{n+1}\right)
	+A_3\limn\left(\sqrt{n+3}-\sqrt{n+1}\right)\\
	&\quad +\ldots
	+A_k\limn\left(\sqrt{n+k}-\sqrt{n+1}\right)=0
\end{align*}
即证。\fin
\end{frame}

\begin{frame}[t]\frametitle{4.用极限定义的函数}
\large

4.求函数$f(x)=\limn\sqrt[n]{1+x^n+\left(\df{x^2}2\right)^n}$
($x\geq 0$)的分段表达式。
\bs

解:\it 由夹逼准则
$$\mbox{原式}
=\max\limits_{x\geq0}\left\{1,x,\frac{x^2}2\right\}=
\left\{\begin{array}{ll}
	1,& x\in[0,1]\\
	x,& x\in(1,2)\\
	\frac{x^2}2,& x\in[2,+\infty)
\end{array}\right.$$
\fin
\end{frame}

\begin{frame}[t]\frametitle{5.和的极限}
\large
5.设$|\lambda|<1$,计算
$\limn(\lambda+2\lambda^2+\ldots+n\lambda^n)$ 

解:\it\small
\begin{align*}
	\lambda+&2\lambda^2+\ldots+n\lambda^n\\
	=&\lambda(1+\lambda+\lambda^2+\ldots+\lambda^{n-1})
	+\ldots+\lambda^{n-1}(1+\lambda)+\lambda^n\\
	=&\df{\lambda}{1-\lambda}\left[(1-\lambda^{n})
	+\lambda(1-\lambda^{n-1})+\ldots+\lambda^{n-2}(1-\lambda^2)
	+\lambda^{n-1}(1-\lambda)\right]\\
	=&\df{\lambda}{1-\lambda}\left[1+\lambda+\lambda^2
	+\ldots+\lambda^{n-1}-n\lambda^{n}\right]\\
	=&\df{\lambda}{(1-\lambda)^2}-\df{\lambda^{n+1}}{(1-\lambda)^2}
	-\df{n\lambda^{n+1}}{1-\lambda}.
\end{align*}
\large
显然$\limn\lambda^{n+1}=0$,以下证明$\limn n\lambda^n=0$。
\end{frame}

\begin{frame}[t]\frametitle{5.和的极限}
\it
以下证明$\limn n\lambda^n=0$。事实上,记$a_n=n\lambda^n$,注意到$\limn\df{n+1}n=1$,故
$$\df{|a_{n+1}|}{|a_n|}=\df{n+1}{n}|\lambda|\to|\lambda|<1,
\;(n\to\infty)$$
由极限的保号性,存在$N\in\mathbb{Z}_+$,使对任意$n>N$,都有
$$\df{|a_{n+1}|}{|a_n|}<1\quad\Rightarrow\quad
|a_{n+1}|<|a_n|,$$
由此可知数列$\{|a_n|\}$当$n$充分大时单调递减有下界,故必收敛,
设其极限为$A$。在递推式$|a_{n+1}|=\df{(n+1)\lambda}{n}|a_n|$
两边同时令$n\to\infty$,可得
$$A=\lambda A\quad\Rightarrow\quad A=0.$$
从而可知$\limn a_n=0$。

至此,可知$\mbox{原式}=\df{\lambda}{(1-\lambda)^2}$。
\fin
\end{frame}

\begin{frame}
	\centering
	\includegraphics[width=\textwidth]{./images/ch01/HWR/notebook.jpg}

	\begin{flushright}
		\color{white}\vspace*{-2cm}
		\Huge\bf Q \& A
	\end{flushright}
\end{frame}