\documentclass[a4paper, openany, punct, adobefonts, UTF8]{ctexbook}%{ctexart}
% \documentclass[a4paper, openany, UTF8, punct, adobefonts]{book}
% \documentclass[a4paper, openany, twocolumn, UTF8, punct,adobefonts]{ctexbook}%{ctexart}
% \documentclass[a4paper,twocolumn,12pt]{article}

%%%%% page layout
% \usepackage[top=1in, bottom=1in, left=1.25in, right=1in]{geometry}
% \usepackage[top=1in, bottom=1in, left=1in, right=1in]{geometry}
% \usepackage[left={3cm},right={3cm},marginparwidth={2cm},
% marginparsep={1em},vmargin={2cm},]{geometry}
\usepackage[left={2cm},right={4cm},marginparwidth={3cm},
	marginparsep={1em},vmargin={2cm}]{geometry}
% \usepackage[top=.6in, bottom=.6in, left=1in, right=1in]{geometry}
\setlength{\parindent}{0pt}

%%%%% TOC with hyperlinks
\usepackage[colorlinks,linkcolor=blue!80!yellow,
	anchorcolor=red,citecolor=green]{hyperref}

%%%%% math settings
\usepackage{amsmath,amsfonts,amssymb,amsthm,bm}
\usepackage{tikz}
\usetikzlibrary{arrows,snakes,backgrounds,petri}
\usepackage{extarrows}%任意长的等号, \xlongequal
\usepackage{color,xcolor,graphics,framed}
\usepackage{graphicx}
\usepackage[justification=centering]{caption} %全局图片标题居中
%\captionsetup[figure]{textfont=it}
\usepackage{subcaption}
%\usepackage[labelfont=bf,textfont=normalfont,singlelinecheck=off,%justification=raggedright]{subcaption}
\usepackage{ulem}	%各种下划线:uline, uuline, uwave, sout, xout
%\usepackage{enumerate}
\usepackage{esint} %any type of integral symbol
% \usepackage{tabularx}
% \usepackage{multirow}
% \usepackage{verbatim}
% \usepackage{listings}
\usepackage{polynom}%多项式的竖式带余除法
\usepackage{overpic}%允许在图片上写文字
\usepackage{wrapfig}%支持插入文字环绕的图片
\usepackage{booktabs}%粗的表格横线
\usepackage{flushend,cuted}
\usepackage{tcolorbox} %各种自定义的盒子
\tcbuselibrary{skins, breakable, theorems}

%%%%% if-then control
\usepackage{ifthen}

%%%%% extended list environments
\usepackage{paralist}


%%%%% Chinese Font Config
% \usepackage{CJKfntef}
\usepackage{fontspec, xunicode, xltxtra}
\usepackage{xeCJK}
%\setmainfont{Times New Roman} 						% 默认的英文字体
\setmainfont{Cambria} 						% 默认的英文字体
\setCJKmainfont[ItalicFont={Adobe Kaiti Std}, 		% 默认的中文字体
	BoldFont={Adobe Heiti Std}]{Adobe Song Std}
\setCJKsansfont{Adobe Heiti Std}					% serif是有衬线字体,sans serif无衬线字体。
\setCJKmonofont{Adobe Fangsong Std}					% 中文等宽字体

%%%%% content visibility control
%用\newif来定义一个判断变量\ifabc,初始值一律为false,
%则除了\ifabc外,还自动生成两个控制序列\abctrue和\abcfalse;
%用\abctrue来设置为true,还可以用\abcfalse来设置为false
\newif\ifvisible
% \visibletrue
% \visiblefalse

%%%%% head and foot setting
% \usepackage{fancyhdr}
% \pagestyle{fancy}
% % \columnsep=10mm
% \fancyhead{}
% \fancyhead[RE]{\leftmark} % 在偶数页的右侧显示章名
% \fancyhead[LE,RO]{\rightmark} % 在奇数页的左侧显示小节名
% % \fancyhead[RE,LO]{}
% % \fancyhead[LE,RO]{~\thepage~} % 在偶数页的左侧,奇数页的右侧显示页码
% \fancyfoot[LE,RO]{\color{gray} NUDT-2017-S3\;\date}
% \renewcommand{\headrulewidth}{2pt}
% \renewcommand{\footrulewidth}{1pt}
% \renewcommand{\headrulewidth}{0pt}
% \usepackage[CJKbookmarks=true,bookmarksnumbered,bookmarksopen,
% colorlinks,linkcolor=blue,anchorcolor=blue,citecolor=green,dvipdfm]{hyperref}
% \excludecomment{student}
% \includecomment{teacher}

%%%%% page watermark
\usepackage[all, scale=10, color=purple!5,
angle=70,contents=NUDT-2018-S3]{background}

%%%%% background color of the page
% \definecolor{myback}{RGB}{204,232,207}
% \pagecolor{yellow!10!white}
% \renewcommand{\baselinestretch}{1.25}