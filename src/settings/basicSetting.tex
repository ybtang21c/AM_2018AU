% !TEX root = ../0-main-LN.tex

\documentclass[a4paper, punct, adobefonts, UTF8]{ctexbook}
%openany 可以在任意页开始新的一章,否则只能在奇数页开始
%punct 对中文标点进行调整
%noindent 段首不缩进

%\documentclass[a4paper, openany, punct, adobefonts, UTF8]{ctexbook}%{ctexart}
% \documentclass[a4paper, openany, UTF8, punct, adobefonts]{book}
% \documentclass[a4paper, openany, twocolumn, UTF8, punct,adobefonts]{ctexbook}%{ctexart}
% \documentclass[a4paper,twocolumn,12pt]{article}

%\usepackage[keeplastbox]{flushend}		%使双栏文本的底端对齐
%\usepackage{cuted}			%支持双栏文本

%%%%% page layout
% \usepackage[top=1in, bottom=1in, left=1.25in, right=1in]{geometry}
% \usepackage[top=1in, bottom=1in, left=1in, right=1in]{geometry}
% \usepackage[left={3cm},right={3cm},marginparwidth={2cm},
% marginparsep={1em},vmargin={2cm},]{geometry}
%\usepackage[left={2cm},right={4cm},marginparwidth={3cm},
%	marginparsep={1em},vmargin={2cm}]{geometry}
\usepackage[left={2cm},right={5cm},marginparwidth={4cm},
	marginparsep={1em},vmargin={2cm}]{geometry}
% \usepackage[top=.6in, bottom=.6in, left=1in, right=1in]{geometry}
%\setlength{\parindent}{0pt}

%%%%% margin note
\usepackage{marginnote}

%%%%% TOC with hyperlinks
\usepackage[colorlinks,linkcolor=blue!80!yellow,
	anchorcolor=red,citecolor=green]{hyperref}

%%%%% math settings
\usepackage{amsmath}		%数学环境,align, align*
\usepackage{amsfonts}		%数学字体,mathbb
\usepackage{amssymb}		%数学公式
%\usepackage{amsthm}			%已包含在amsmath中
%\usepackage{mathrsfs}		%数学花体
\usepackage{extarrows}		%\xlongequal
\usepackage{bm}				%bm

\usepackage{tikz}
%\usetikzlibrary{arrows,snakes,backgrounds,petri}

%\usepackage{color}
\usepackage{xcolor}

\usepackage{graphics}
\usepackage{graphicx}		%definecolor, colorbox, fcolorbox
\usepackage{wrapfig}		%文字环绕的图片

\usepackage{framed}			%support shaded environment

\usepackage[justification=centering]{caption} %全局图片标题居中
\captionsetup[figure]{font={small,it},labelfont={up}}
\usepackage{subcaption}
%\usepackage[labelfont=bf,textfont=normalfont,singlelinecheck=off,%justification=raggedright]{subcaption}

\usepackage{ulem}			%各种下划线:uline, uuline, uwave, sout, xout
%\usepackage{enumerate}
%\usepackage{esint}			%any type of integral symbol
% \usepackage{tabularx}
% \usepackage{multirow}
% \usepackage{verbatim}
% \usepackage{listings}
\usepackage{polynom}		%多项式的竖式带余除法
%\usepackage{overpic}		%允许在图片上写文字
%\usepackage{wrapfig}		%支持插入文字环绕的图片
\usepackage{booktabs}		%粗的表格横线

\usepackage{tcolorbox} 		%各种自定义的盒子
\tcbuselibrary{skins, breakable, theorems}

%%%%% extended list environments
\usepackage{paralist}		%控制段间距等

%%%%% fontspec
\usepackage{fontspec}
\setmainfont[BoldFont={Georgia Bold},
	ItalicFont={Georgia Italic},
	BoldItalicFont={Georgia Bold Italic}
	]{Georgia}

%%%%% XeCJK
\usepackage{xeCJK}
\setCJKmainfont[ItalicFont={Adobe Kaiti Std},
	BoldFont={Adobe Heiti Std}]{Adobe Song Std}
\setCJKsansfont{Adobe Fangsong Std}
\setCJKmonofont{Adobe Heiti Std}
\setCJKmathfont{Cambria Math}
%%%%% font families
%-----------------------xeCJK下设置中文字体------------------------------%
%常用字体
\setCJKfamilyfont{song}{Adobe Song Std}				%Adobe宋 \song
\newcommand{\song}{\CJKfamily{song}}                
      
\setCJKfamilyfont{fsong}{Adobe Fangsong Std}			%adobe仿宋 \fsong
\newcommand{\fsong}{\CJKfamily{fsong}}

\setCJKfamilyfont{kai}{Adobe Kaiti Std}				%Adobe楷体 \kaiti
\newcommand{\kaiti}{\CJKfamily{kai}}

\setCJKfamilyfont{hei}{Adobe Heiti Std}				%Adobe黑体 \heiti
\renewcommand{\heiti}{\CJKfamily{hei}}

\setCJKfamilyfont{hwzs}{STZhongsong}				%华文中宋 \hwzs
\newcommand{\hwzs}{\CJKfamily{hwzs}}

\setCJKfamilyfont{yh}{Microsoft YaHei}				%微软雅黑 \msyh
\newcommand{\msyh}{\CJKfamily{yh}}

%其他字体
\setCJKfamilyfont{hwfs}{STFangsong}				%华文仿宋  hwfs
\newcommand{\hwfs}{\CJKfamily{hwfs}}

\setCJKfamilyfont{hwxh}{STXihei}					%华文细黑  hwxh
\newcommand{\hwxh}{\CJKfamily{hwxh}}

\setCJKfamilyfont{hwl}{STLiti}						%华文隶书  hwl
\newcommand{\hwls}{\CJKfamily{hwl}}

\setCJKfamilyfont{hwxw}{STXinwei}					%华文新魏  hwxw
\newcommand{\hwxw}{\CJKfamily{hwxw}}

\setCJKfamilyfont{hwxk}{STXingkai}					%华文行楷  hwxk
\newcommand{\hwxk}{\CJKfamily{hwxk}}

\setCJKfamilyfont{hwhp}{STHupo}					%华文琥珀   hwhp
\newcommand{\hwhp}{\CJKfamily{hwhp}}

\setCJKfamilyfont{wawati}{Wawati SC}				%娃娃体 wawati
\newcommand{\wawati}{\CJKfamily{wawati}}

%------------------------------设置字体大小------------------------%
\newcommand{\chuhao}{\fontsize{42pt}{\baselineskip}\selectfont}     %初号
\newcommand{\xiaochuhao}{\fontsize{36pt}{\baselineskip}\selectfont} %小初号
\newcommand{\yihao}{\fontsize{28pt}{\baselineskip}\selectfont}      %一号
\newcommand{\erhao}{\fontsize{21pt}{\baselineskip}\selectfont}      %二号
\newcommand{\xiaoerhao}{\fontsize{18pt}{\baselineskip}\selectfont}  %小二号
\newcommand{\sanhao}{\fontsize{15.75pt}{\baselineskip}\selectfont}  %三号
\newcommand{\sihao}{\fontsize{14pt}{\baselineskip}\selectfont}%     四号
\newcommand{\xiaosihao}{\fontsize{12pt}{\baselineskip}\selectfont}  %小四号
\newcommand{\wuhao}{\fontsize{10.5pt}{\baselineskip}\selectfont}    %五号
\newcommand{\xiaowuhao}{\fontsize{9pt}{\baselineskip}\selectfont}   %小五号
\newcommand{\liuhao}{\fontsize{7.875pt}{\baselineskip}\selectfont}  %六号
\newcommand{\qihao}{\fontsize{5.25pt}{\baselineskip}\selectfont}    %七号

%%%%% head and foot setting
 \usepackage{fancyhdr}
 \pagestyle{fancyplain}
 %\renewcommand{\chaptermark}[1]{\markboth{\chaptername .\;\ #1}{}}
 %\renewcommand{\chaptermark}[1]{\markboth{\chaptername .\;\ #1}{}}
 %\renewcommand{\sectionmark}[1]{\markright{\thesection \ #1}{}}
 \fancyhf{}
 \fancyhead[RE,LO]{\fangsong\leftmark}
 \fancyhead[LE,RO]{\fangsong\rightmark}
 \fancyhead[CE,CO]{} %
 \fancyfoot[LE,RO]{\color{gray} AM-NUDT-2018-S3,
 \number\day-\number\month-\number\year}
 \fancyfoot[LO,RE]{\fangsong -\,\thepage\,-}
% \renewcommand{\headrulewidth}{2pt}
% \renewcommand{\footrulewidth}{1pt}
% \renewcommand{\headrulewidth}{0pt}
% \usepackage[CJKbookmarks=true,bookmarksnumbered,bookmarksopen,
% colorlinks,linkcolor=blue,anchorcolor=blue,citecolor=green,dvipdfm]{hyperref}
% \excludecomment{student}
% \includecomment{teacher}
\clearpage{\pagestyle{empty}\cleardoublepage}

%%%%% page watermark
\usepackage[all, scale=10, color=purple!5,
angle=70,contents=NUDT-2018-S3]{background}

%%%%% background color of the page
% \definecolor{myback}{RGB}{204,232,207}
% \pagecolor{yellow!10!white}
% \renewcommand{\baselinestretch}{1.25}

%%%%% if-then control
\usepackage{ifthen}
%%%%% content visibility control
%用\newif来定义一个判断变量\ifabc,初始值一律为false,
%则除了\ifabc外,还自动生成两个控制序列\abctrue和\abcfalse;
%用\abctrue来设置为true,还可以用\abcfalse来设置为false
%\ifabc和\fi之间的内容由\ifabc来控制是否出现
\newif\ifvisible
\newif\ifanswer  % if the answer of homework
\newif\ifhint % if the hint to exercise
\newif\ifnottest % if its a test
% \visibletrue
% \visiblefalse