% !TEX root = ../0-main-LN.tex

\setcounter{chapter}{1}

\chapter{导数与微分}

导数刻画了函数关于自变量的变化率,在现实问题中有很多对应的实例,例如:
速度、密度、压强、出生率、死亡率、曲线斜率,等等。导数的引入,极大
增强了用数学工具刻画现实问题的能力,其作用可以说是飞跃式的。

\section{导数的概念}

\subsection{函数在一点处的导数}

\begin{thx}
	设函数$y=f(x)$在$x_0$的某领域内有定义,若
	$$\limdx\df{f(x_0+\dx)-f(x_0)}{\dx}$$
	存在, 则称{\bf $f(x)$在$x_0$处可导},该极限称为{\bf $f(x)$在$x_0$处的导数}, 记为$f'(x_0)$,或以下符号之一
	$$\left.\df{\d y}{\d x}\right|_{x=x_0},\quad\left.\df{\d
	}{\d x}y\right|_{x=x_0},\quad \left.\df{\d }{\d x}f(x)\right|_{x=x_0}\quad
	y'_x|_{x=x_0},\quad \dot{y}(x_0)$$
\end{thx}

\begin{shaded}
	Newton把连续的量统称为{\it 流量}(fluent),
	把变化率称为{\it 流数}(fluxion),
	把自变量理解为时间,时间的一个微小改变量称为
	{\it 瞬}(moment)。他将流量$x$的流数
	记为$\dot{x}$,二阶流数记为$\ddot{x}$,
	这种符号有时在{\it 动力系统}(以时间为自变量的系统)
	的讨论中还会使用,但更多的时候我们都使用现在的这套符号,
	它们(包括积分的符号)
	都是由Leibniz最初引入的。

	Newton的导数符号形如$\dot{y}$(一阶导数)或$\ddot{y}$
	(二阶导数)。这类符号存在两个方面的局限性,一是不利于
	向更高的阶数扩展(想象一下$y$的头上打很多点),二是
	它没有从形式上展示出导数和微分(无穷小)之间微妙的联系。

	\pss{\centering
	\includegraphics[width=0.8\marginparwidth]
	{./images/ch02/Gottfried_Wilhelm_Leibniz,_Bernhard_Christoph_Francke.jpg}\\
	\href{https://en.wikipedia.org/wiki/Gottfried_Wilhelm_Leibniz}{G. Leibniz(1646-1716)}}
	相对而言,Leibniz在符号的设计上则花了更多的精力,今天我们
	常用的微分符号$\d x,\d y$,求导运算$\df{\d y}{\d x}$,
	$\df{\d^n y}{\d x^n}$,$y'(x)$,$y^{(n)}(x)$,
	以及定积分$\dint_a^bf(x)\d x$据说都是他的发明。
	这些符号已经被证明是描述微积分系统的最简洁、直观和
	有力的工具。

	有人说,Newton研究微积分是更偏重于从物理的角度入手,
	是经验的、具体的,而Leibniz则更像一个数学家,
	建立了微积分的规范,也即法则和公式的系统。物理学家
	为了研究的便利,强调简洁,却一定程度上制约了符号自身的扩展,
	数学家则更看重数学自身体系的严谨一致,同时也兼顾简洁和便利。
\end{shaded}

在形如$\df{\d y}{\d x}$和$\df{\d}{\d x}$的导数符号中,$\df{\d}{\d x}$可以
被理解为所谓的{\it 求导算子},表示对其右方的函数求导。求导算子的优先级高于四则运算
和一般的函数运算。

\subsubsection{导数的意义}

从导数的定义不难看出,导数本质上是一种{\it 变化率},严格地说,是某一点附近的
函数平均变化率的极限,在现实的问题中,有大量这种平均变化率的例子,
但有了导数,我们把这种变化率精确地与某个点(位置、时刻)对应了起来。

\begin{itemize}
  \setlength{\itemindent}{1cm}
  \item {\it 切线斜率:}
  $$k(x_0)=\lim\limits_{x\to x_0}\df{f(x)-f(x_0)}{x-x_0}$$
  \item {\it 瞬时速度:}
  \ps{在现实中,无法测量所谓的瞬时速度,因此它完全可以说是一个抽象概念}
  $$v(t_0)=\lim\limits_{t\to t_0}\df{S(t)-S(t_0)}{t-t_0}$$
  \item {\it 出生率:}
  $$\gamma(s_0)=\lim\limits_{s\to s_0}\df{[s_0,s]\mbox{内出生的人口总数}}{s-s_0}$$
  \item {\it 经济的增长率:}
  $$S_e=\lim\limits_{y\to y_0}\df{y\mbox{年的经济总量}
  -y_0\mbox{年的经济总量}}{y-y_0}$$
\end{itemize}

导数是函数关于自变量的变化率,也就是当自变量发生变化时,函数值发生的相应变化
与之的比率。变化率的另一种理解是与自变量的单位变化量对应的函数值的改变量,
例如:速度是单位时间内的位移,密度是单位体积对应的质量。

% {\bf 导数:函数关于自变量的变化率!}

\begin{figure}[h]
	\centering
	\includegraphics[width=0.5\textwidth]{./Images/Ch02/cut2Penp.pdf}
	\caption{导数的几何意义:当$\Delta x\to0$时,
	割线的极限位置(切线)的斜率}
	\label{fig:cut2Penp}
\end{figure}
	
如图\ref{fig:cut2Penp},从几何上看,
$k=\df{\Delta y}{\Delta x}$可以理解为过$M_0(x_0,f(x_0))$和
$M_1(x_0+\Delta x,f(x_0+\Delta x))$的割线斜率,随着$\Delta x\to0$
(也即$M_1$向$M_0$不断靠近),
该割线可能逐渐趋于某个极限位置(这个位置也就是我们通常所说的{\kaishu 切线}),
因此,导数的几何意义可以理解为:函数曲线在给定点处的切线斜率。

根据导数的几何意义,已知函数$f(x)$在点$x_0$处可导,可以得到曲线$y=f(x)$
在该点的切线和法线方程:
\begin{thx}
	\begin{itemize}
% 	  \setlength{\itemindent}{1cm}
	  \item {\it 切线:\quad $y=f(x_0)+f'(x_0)(x-x_0)$}
	  \item {\it 法线:\quad $y=f(x_0)-\df1{f'(x_0)}(x-x_0)\quad(f'(x_0)\ne 0)$}
	\end{itemize}
\end{thx}

\egz 若抛物线$y=x^2$上有三个不同号点处的法线交于一点,这三个点的横坐标需要满足什么条件?

解: 设三个点的横坐标分别是$a_1,a_2,a_3$。

情形一:若某个$a_i=0$(不妨为$a_1=0$),显然由对称性可知,必有$a_2=-a_3$。

情形二:若$a_1,a_2,a_3$均非零,我们可以给出$a_i(i=1,2,3)$处的法线方程
$$y=a_i^2-\df1{2a_i}(x-a_i)$$
化简后可得
$$\df{a_1+a_2}{a_3}=\df{a_1+a_3}{a_2}=\df{a_2+a_3}{a_1}$$
进而
$$\df{a_1+a_2+a_3}{a_3}=\df{a_1+a_2+a_3}{a_2}=\df{a_1+a_2+a_3}{a_1}$$
显然$a_,a_2,a_3$相互不同,故必有$a_1+a_2+a_3=0$。
\fin

\bs
{\bf 思考:}函数在一点可导等价于其图像在该点存在切线吗?

\ifhint
答:不等价。可导则必存在切线,反之不然。若切线是铅直方向的,则导数没有意义!
例如:函数$f(x)=\sqrt[3]x$在点$x=0$不可导(导数为$\infty$),
但存在切线(铅直方向)。
\fi

\subsubsection{导数存在的条件}

为了讨论的方便,我们有时会使用“{\it 左(右)导数}”的概念,对应于导数定义中分别取
左(右)导数的情形,记为{$f'_-(x_0)$}和{$f'_+(x_0)$},
显然由导数的定义,$f(x)$在$x_0$可导,当且仅当在该点的左、右导数存在且相等。

不难证明,函数在一点连续是在该点可导的一个必要条件。

\begin{thx}
	{\bf 定理:}$f(x)$在一点可导,则一定在该点连续。
\end{thx}

证:设$f(x)$在$x_0$可导,导数为$A$,则
\begin{align*}
	\limx{x_0}[f(x)-f(x_0)]
	&=\limx{x_0}\df{f(x)-f(x_0)}{x-x_0}\cdot(x-x_0)\\
	&=\limx{x_0}\df{f(x)-f(x_0)}{x-x_0}\limx{x_0}(x-x_0)\\
	&=A\cdot 0=0,
\end{align*}
从而可知$\limx{x_0}f(x)=f(x_0)$,也即$f(x)$在$x_0$连续。
\fin

\bs
{\bf 课堂练习:} 假设$f\,'(x_0)$存在,则
\begin{enumerate}[(1)]
  \setlength{\itemindent}{1cm}
  \item $\limdx\df{f(x_0-\Delta x)-f(x_0)}{\Delta
  x}=$ \underline{\quad\quad} 
  \item $\lim\limits_{h\to 0}\df{f(x_0+2h)-f(x_0)}{h}=$
   \underline{\quad\quad} 
  \item $\lim\limits_{h\to 0}\df{f(x_0+h)-f(x_0-h)}{h}=$ 
  \underline{\quad\quad}
\end{enumerate}
以上哪个极限存在,与$f(x)$在$x_0$可导等价?

\ifhint
答:
\begin{enumerate}[(1)]
  \setlength{\itemindent}{1cm}
  \item $\limdx\df{f(x_0-\Delta x)-f(x_0)}{\Delta
  x}=$ \underline{\quad{$-f\,'(x_0)$}\quad} 
  \item $\lim\limits_{h\to 0}\df{f(x_0+2h)-f(x_0)}{h}=$
   \underline{\quad{$2f\,'(x_0)$}\quad} 
  \item $\lim\limits_{h\to 0}\df{f(x_0+h)-f(x_0-h)}{h}=$ 
  \underline{\quad{$2f\,'(x_0)$}\quad}
\end{enumerate}
只有(3)与可导的定义不等价,因为其中没有提到$f(x_0)$,换言之,
即便$f(x_0)$没有定义,(3)也可能成立,但这样一来导数的定义就没有意义了!
\fi

\bs
\egz 确定常数$a,b$的值,使得函数
$$f(x)=\left\{\begin{array}{ll}ax+b,& x>0\\
e^x,& x\leq 0\end{array}\right.$$
在$x=0$可导。

解:$f(x)$在$x=0$可导,则必在$x=0$连续,于是由
$$f(x+0)=f(x-0)\quad\Rightarrow\quad b=1.$$
又
\begin{align*}
	f'_+(0)&=\limx{0^+}\df{ax+b-1}{x}=a,\\
	f'_-(0)&=\limx{0^-}\df{e^x-e^0}{x}=1,
\end{align*}
故$a=1$。\fin

{\baa 注意:计算分段定义的函数的导数,在分段的点上
的左右导数要分别用定义计算!}

\bs
\egz 已知$f(x)$在$x=0$连续,且$\limx{0}\df{f(x)}x=A$,
证明$f(x)$在$x=0$可导,并求$f'(0)$。

证:由$f(x)$在$x=0$连续,
$$f(0)=\limx0f(x)=\limx0\df{f(x)}x\cdot x
=\limx0\df{f(x)}x\cdot\limx0x=A\cdot 0=0,$$
从而
$$A=\limx{0}\df{f(x)}x=\limx{0}\df{f(x)-f(0)}{x-0}=f'(0),$$
由此可知$f(x)$在$x=0$可导,且$f'(0)=A$。
\fin

\bs
当然,连续并不是可导的充分条件,例如$f(x)=|x|$在$x=0$处连续,但不可导,
其左右导数分别为$\pm 1$。

\bs
{\bf 讨论:}函数在一点不可导,可能有哪些不同的形态?

\ifhint
提示:如图\ref{fig:nonDeriv}.
\begin{figure}[h]
	\centering
	\begin{subfigure}[t]{0.3\textwidth}
		\centering
		\includegraphics[width=\textwidth]
		{./images/ch02/nonCont.pdf}
		\caption{不连续的点}
	\end{subfigure}\;\;
	\begin{subfigure}[t]{0.3\textwidth}
		\centering
		\includegraphics[width=\textwidth]
		{./images/ch02/vertTouch.pdf}
		\caption{切线为铅直方向的点}
	\end{subfigure}\;\;
	\begin{subfigure}[t]{0.3\textwidth}
		\centering
		\includegraphics[width=\textwidth]
		{./images/ch02/sharpCor.pdf}
		\caption{“尖点”}
	\end{subfigure}
	\caption{$f(x_0)$有定义但不可导的各种情况}
	\label{fig:nonDeriv}
\end{figure}
\fi

\bs
\egz 函数$f(x)=x^2D(x)$在$x=0$可导,但在其他地方都不连续。

提示:参考$xD(x)$在$x=0$以外的点都不存在极限,可以证明
$f(x)=x^2D(x)$在$x=0$之外都不连续。

又注意到$0\leq x^2D(x)\leq x^2$,故对任意$x\ne 0$,
$$0\leq \df{x^2D(x)}{x}\leq x,$$
从而由夹逼定理,可知
$$\limx{0}\df{f(x)-f(0)}{x-0}=\limx{0}\df{x^2D(x)}{x}=0,$$
即证。
\fin

\begin{shaded}
	连续但不可导最极端的情况是如下的{\kaishu Weierstrass函数},
	可以证明(涉及函数项级数的一些性质,在此不作证明)它在处处连续但处处不可导:	
	$$W(x)=\sum\limits_{n=0}^{\infty}a^n\cos(b^n\pi x),$$
	其中$a\in(0,1)$,$b$为正的奇数,满足$ab>1+\df32\pi$
	\begin{center}
		\resizebox{!}{6cm}{\includegraphics
		{./images/ch01/WeierstrassFunction.png}}
		
		\kaishu Weierstrass函数具有某种“分形”的特征,对其任何一个微小的局部
		进行放大,都会呈现处相似的锯齿状形态,它是一个真正在任一段上都不光滑的函数。
	\end{center}

	\pss{\centering
		\includegraphics[width=0.8\marginparwidth]
		{./images/ch02/calculusGallery.jpg}\\
		\href{https://book.douban.com/subject/4904723/}
		{《微积分的历程》}
	}

	这个反例的引入对数学界造成了巨大的震动,
	但同时也产生了无可替代的巨大意义。
	《微积分的历程》一书中(P.169)
	的一段评述可以给我们启发:{\it 在持续不断的起伏中,数学家们建立起
	雄伟的理论体系,然后寻找足以揭示他们的思想界限的恰当反例。
	这种理论与反例的对照
	成为正确推理的引擎,凭借这种工具,数学得以进步。因为{ 我们唯有知道某些特性是如何
	丧失的,方能了解它们是怎样发挥作用的。同样,
	我们唯有认清直觉是如何把人引入歧途
	的,方能如实地评价推理的威力。}}
\end{shaded}

{\bf 讨论:}
\begin{enumerate}[(1)]
  \setlength{\itemindent}{1cm}
  \item 若$f(x),g(x)$在$x_0$均不可导,是否$f(x)+g(x),$ $f(x)g(x)$必不可导?
  
  \ifhint\quad ({$\times$})反例:$f(x)=D(x)$,$g(x)=1-D(x)$。\fi
  \item 若对任意$x\in (a,b)$,恒有$f(x)<g(x)$,且$f(x),g(x)$均在$(a,b)$内
  可导,问是否必有$f\,'(x)<g'(x)$? 
  
  \ifhint\quad({$\times$})反例:$f(x)=\sin x$,$g(x)=2$。 \fi
  \item $f(x)$可导,则$|f(x)|$可导?反之呢?
  
  \ifhint\quad({$\times$})反例:都错误,$f(x)=x$,$|f(x)|=|x|$在$x=0$不可导;反之,考虑$f(x)=D(x)-\df12$。\fi
  \item 若$f(x)$在$\mathbb{R}$上可导,且$\limx{+\infty}f(x)=\infty$,是否
  必有$\limx{+\infty}f\,'(x)=\infty$?反之呢? 
  
  \ifhint\quad({$\times$})反例:$f(x)=x$。\fi
  \item 若$f(x)$在$(a,b)$内可导,且$\limx{a^+}f(x)=\infty$,是否
  必有$\limx{a^+}f\,'(x)=\infty$? 反之呢?
  
  \ifhint\quad({$\times$})反例:$f(x)=\df1x+\sin\df1x$。 \fi
  \item 若$f(x)$可导且为奇(偶)函数,则$f\,'(x)$也有奇偶性? 
  (相应的$f'(0)$有什么特点?偶函数,$f'(0)=0$)

  \ifhint\quad({$\surd$})事实上,设对任意的$x$均有$f(-x)=f(x)$,则
  \begin{align*}
  	f'(-x)&
  =\lim\limits_{\Delta x\to 0}\df{f(-x+\Delta x)-f(-x)}{\Delta x}
  =\lim\limits_{\Delta x\to 0}\df{-f(x-\Delta x)+f(x)}{\Delta x}\\
  &=\lim\limits_{\Delta x\to 0}\df{f(x-\Delta x)-f(x)}{-\Delta x}
  =f'(x),
  \end{align*}
  也即奇函数的导函数必为偶函数。同理可证,偶函数的导函数必为奇函数。
  \fi
  \item 若$f(x)$可导且为周期函数,则$f\,'(x)$也是周期函数? 

  \ifhint\quad({$\surd$})证明略。
  \fi
\end{enumerate}

% \egz 设$f(x)=\sum\limits_{i=1}^na_i\sin
% ix$,其中$a_i(i=1,2,\ldots,n)$为常数,且对任意$x\in\mathbb{R}$, 
% $|f(x)|\leq |\sin x|$,证明:
% $$\left|a_1+2a_2+\ldots+na_n\right|\leq 1$$
% 
% 提示:$f(0)=0$,于是
% $$|f'(0)|=\limx{0}\df{|f(x)|}{|x|}\leq\limx{0}\df{|\sin x|}{|x|}=1$$
% 事实上$f'(0)=a_1+2a_2+\ldots+na_n$.

\bs
{\bf 思考:}
\begin{enumerate} 
  \setlength{\itemindent}{1cm}
  \item $f'(x_0)$和$[f(x_0)]'$,$f'(x)$和$[f(x)]'$有何异同?

  \ifhint
  \quad 提示:$f'(x_0)$表示函数在$x_0$处的导数;$[f(x_0)]'$表示对
  $f(x_0)$求导,显然结果为$0$;$f'(x)$表示$f(x)$的导函数,$[f(x)]'$
  也是。
  \fi
  \item $f'_+(x_0)$和$f'(x_0+0)$有何区别?
  \ps{有些书上也把$f'(x_0+0)$记为$f'(x_0^+)$}

  \ifhint
  \quad 提示:$f'_+(x_0)$表示$f(x)$在$x_0$处的右导数;
  $f'(x_0+0)$表示$f(x)$的导函数$f'(x)$在$x_0$处的右极限。
  \fi
  \item $f(x)=g(x)$可推出$f'(x)=g'(x)$?反之呢?

  \ifhint
  \quad 提示:反之不成立。例如$f(x)=x$和$g(x)=x+1$,导函数相同。
  事实上,可以证明:{\bf 导函数相同的函数之间只相差一个常数},也即
  若$f'(x)=g'(x)$,则必存在某个$C\in\mathbb{R}$,使得
  $f(x)=g(x)+C$.
  \fi
  \item 圆的面积$S(r)$关于直径的导数$S'(r)=l(r)$为圆周长,
  球体积关于直径的导数为表面积,如何解释?类似的,
  (定长)圆柱体的体积关于截面半径的导数
  等于其与侧面积?矩形的体积关于各边长的导数等于其对应的截面积?

  \ifhint
  \quad 提示:圆的周长是其面积关于半径的变化率。
  \fi
\end{enumerate}

\subsection{导函数}

有了函数在一点的导数概念,可以很容易地将其扩展到所谓的{\it 导函数},也即由函数
在不同点处的导数值所构成的函数。为了今后的讨论方便,不加证明地引入如下定理:

\begin{thx}
	{\bf 初等函数的可导性:}所有初等函数在其定义域内均是处处可导的。
\end{thx}

事实上,经过本章稍后的讨论,我们会发现,初等函数的导函数都是初等函数。
但是,需要提醒一句的是,并非只有初等函数的导函数才是初等函数。例如没有一个
初等函数的导函数是$e^{x^2}$或者$\df{\sin x}x$。

接下来我们的主要任务就是给出所有初等函数的导函数。

利用导数的定义和基本的极限运算,可以很容易地求出一些常用初等函数的导函数:

\begin{thx}
	{\bf 基本初等函数的求导公式I}
	\begin{enumerate}[(1)]
% 	  \setlength{\itemindent}{1cm}
	  \item $(C)'=0\;(C\mbox{为常数})$ 
	  \item $(x^n)'=nx^{n-1}\;(n\in\mathbb{Z},n\ne
	  0)$ 
	  \item $(e^x)'=e^x$ 
	  \item $(\ln|x|)'=\df1x$ \ps{注意绝对值符号!!}
	  \item $(\sin x)'=\cos x=\sin\left(x+\df{\pi}2\right)$
	  \item $(\cos x)'=-\sin x=\cos\left(x+\df{\pi}2\right)$
	\end{enumerate}
\end{thx}

证:(1)
\begin{align*}
	(C)'=\lim\limits_{\Delta x\to0}\df{C-C}{\Delta x}=0.
\end{align*}

(2)
\begin{align*}
	(x^n)'
	&=\lim\limits_{\Delta x\to0}\df{(x+\Delta x)^n-x^n}{\Delta x}\\
	&=\lim\limits_{\Delta x\to0}
	\df{C_n^1x^{n-1}\Delta x+C_n^2x^{n-2}(\Delta x)^2+\ldots
	+C_{n}^{n-1}x(\Delta x)^{n-1}+(\Delta x)^n}{\Delta x}\\
	&=\lim\limits_{\Delta x\to0}\left[C_n^1x^{n-1}
	+C_n^2x^{n-2}\Delta x+\ldots
	+C_{n}^{n-1}x(\Delta x)^{n-2}+(\Delta x)^{n-1}\right]\\
	&=nx^{n-1}.
\end{align*}

(3)
\begin{align*}
	(e^x)'=\lim\limits_{\Delta x\to0}
	\df{e^{x+\Delta x}-e^x}{\Delta x}
	=\lim\limits_{\Delta x\to0}\df{e^x(e^{\Delta x}-1)}{\Delta x}
	=e^x.
\end{align*}

(4)若$x>0$
\begin{align*}
	(\ln |x|)'=(\ln x)'=\lim\limits_{\Delta x\to0}
	\df{\ln(x+\Delta x)-\ln x}{\Delta x}
	=\lim\limits_{\Delta x\to0}
	\df{\ln\left(1+\frac{\Delta x}x\right)}{x\frac{\Delta x}x}
	=\df1x.
\end{align*}
注意到$\ln|x|$为偶函数,故其导函数必为奇函数。故当
$x<0$时,亦有$(\ln|x|)'=\frac 1x$.

(5)
\begin{align*}
	(\sin x)'
	&=\lim\limits_{\Delta x\to0}
	\df{\sin(x+\Delta x)-\sin x}{\Delta x}
	=\lim\limits_{\Delta x\to0}
	\df{2\sin\frac{\Delta x}2\cos\left(x+\frac{\Delta x}2\right)}{\Delta x}\\
	&=\lim\limits_{\Delta x\to0}
	\df{\sin\frac{\Delta x}2}{\frac{\Delta x}2}
	\lim\limits_{\Delta x\to0}
	\cos\left(x+\frac{\Delta x}2\right)
	=\cos x.
\end{align*}

(6)与(5)同理,略。
\fin

\begin{ext}
	{\centering\bf 习题2.1}
	
	\begin{enumerate}  
% 	  \item 确定$a$的值,使$y=ax^2$与$y=\ln x$相切。
	  \item 已知$f(x)=\left\{\begin{array}{ll}
		2e^x+b,& x\leq0\\ ax+\sin x,& x> 0
		\end{array}\right.$
		试确定$a,b$的值,使得$f(x)$在$x=0$处可导。
% 	  \item 求曲线$y=\cos x$在$\left(\df{\pi}3,\df12\right)$处的切线和
% 	  法线方程。
	  \item 证明:曲线$xy=a^2$上任一点处的切线与两坐标轴构成的三角形面积不变。
	  \item 讨论函数
	  	$y=\left\{\begin{array}{ll}
	    	x^2\sin\df1x,& x\ne0;\\ 0, & x=0.
	    \end{array}\right.$
	  在$x=0$处的连续性、可导性以及导函数的连续性。
	  \item 设对任意$x\in\mathbb{R}$,均有$f(x+2)=f(x)$,已知$f'(0)=1$,
	  证明$f(x)$在$x=2$可导,并求$f'(2)$。
	  \item 已知曲线$y=f(x)$和曲线$y=\sin x$在原点相切(即二者的切线相同),
	  求$\limx0\df{f(3x)}x$。
% 	  \begin{enumerate}[(1)]
% 	    \item 
% 	    \item $y=\left\{\begin{array}{ll}
% 	    	x^2\sin\df1x,& x\ne0;\\ 0, & x=0.
% 	    \end{array}\right.$
% 	  \end{enumerate}
	  \item 已知$f'(a)f(a)\ne 0$,求
	  $\limx0\left[\df{f(a+x)}{f(a)}\right]^{\frac1{\sin x}}.$
	  \item 已知函数$g(x)$在$x=a$连续,问函数$f(x)=|(x-a)|g(x)$
	  在$x=a$是否可导?若可导,证明之;若不可导,讨论增加什么样的条件可以使之可导。
	  利用以上讨论的结果,判断$f(x)=(x^2-4)|x^2+3x+2|$有几个不可导的点。
	  \item 设对任意$x,y\in\mathbb{R}$,有
	  $$f(x+y)=f(x)+f(y)+x^2y+xy^2,$$
	  且当$x\to0$时$f(x)$与$x$是等价无穷小,证明$f(x)$处处可导,并求其导函数。
	\end{enumerate}
\end{ext}

\section{函数的求导法则}

上一节给出了一些基本初等函数的导函数,但还不是全部。
此外,对于各种复合函数、反函数等,该如何求导,都是需要
进一步研究的问题。

利用导数的定义求导函数,因为涉及到极限的运算,过程较为繁琐。
本节中,我们将通过介绍基本的求导法则,进一步简化求导的过程,
为计算各种形式的函数的导数提供便利。

\subsection{四则运算的求导法则}

\begin{thx}
	{\bf 四则运算求导公式:}设$u(x),v(x)$均在$x$可导,则
	\begin{enumerate}[(1)]
% 	  \setlength{\itemindent}{1cm}
	  \item $[u(x)\pm v(x)]'=u'(x)\pm v'(x)$ 
	  \item $[u(x)v(x)]' =u'(x)v(x)+u(x)v'(x)$ 
	  \item $\left[\df{u(x)}{v(x)}\right]'
	  =\df{u'(x)v(x)-u(x)v'(x)}{v^2(x)}\;\;(\mbox{假设}v(x)\ne 0)$
	\end{enumerate}
\end{thx}

证:(1)
\begin{align*}
	[u(x)\pm v(x)]'
	&=\lim\limits_{\Delta x\to 0}
	\df{[u(x+\Delta x)\pm v(x+\Delta x)]-[u(x)\pm v(x)]}{\Delta x}\\
	&=\lim\limits_{\Delta x\to 0}
	\df{u(x+\Delta x)-u(x)}{\Delta x}
	\pm\lim\limits_{\Delta x\to 0}
	\df{v(x+\Delta x)-v(x)}{\Delta x}\\
	&=u'(x)\pm v'(x).
\end{align*}

(2)
\begin{align*}
	[u(x)v(x)]'
	&=\lim\limits_{\Delta x\to 0}
	\df{u(x+\Delta x)v(x+\Delta x)-u(x)v(x)}{\Delta x}\\
	&=\lim\limits_{\Delta x\to 0}
	\df{[u(x+\Delta x)-u(x)]v(x+\Delta x)-
	u(x)[v(x+\Delta x)-v(x)]}{\Delta x}\\
	&=\lim\limits_{\Delta x\to 0}v(x+\Delta x)
	\lim\limits_{\Delta x\to 0}\df{u(x+\Delta x)-u(x)}{\Delta x}\\
	&\quad +u(x)\lim\limits_{\Delta x\to 0}
	\df{v(x+\Delta x)-v(x)}{\Delta x}\\
	&=u'(x)v(x)+u(x)v'(x).
\end{align*}

(3)注意到$\frac uv=u\frac1v$
\ps{有时候为了书写简便,在自变量明确的情况下,可以省略自变量}
,故只需证明
$$\left[\frac1{v(x)}\right]'=-\df1{v^2(x)},$$
然后利用以上的乘法的求导公式即得
$$\left(\frac uv\right)=u'\frac1v+u\left(\frac 1v\right)'
=\df{u'v-uv'}{v^2}.$$
注意到
\begin{align*}
	\left[\frac{1}{v(x)}\right]'
	&=\lim\limits_{\Delta x\to 0}
	\df{\frac1{v(x+\Delta x)}-\frac1{v(x)}}{\Delta x}\\
	&=\lim\limits_{\Delta x\to 0}
	\df{v(x)-v(x+\Delta x)}{v(x+\Delta x)v(x)\Delta x}\\
	&=-\lim\limits_{\Delta x\to 0}\frac1{v(x+\Delta x)v(x)}
	\lim\limits_{\Delta x\to 0}\df{v(x)-v(x+\Delta x)}{\Delta x}\\
	&=-\df1{v^2(x)}.
\end{align*}
即证。\fin

\bs
{\bf 思考:}自行推导$(uvw)'$的公式,并给出类似函数的求导规律。

\ifhint
提示:
$$(uvw)'=u'vw+uv'w+uvw'.$$
\fi

\bs
{\bf 课堂练习} 计算以下函数的导函数
\begin{enumerate}[(1)]
  \setlength{\itemindent}{1cm}
  \item $f(x)=2x^3+3x-4x+5-\df 6x$ 
  \item $f(x)=e^x\sin x$ 
  \item $f(x)=\df{x-1}{x+1}$ 
  \item $f(x)=\df 1{\ln x}$ 
\end{enumerate}

\bs
利用四则运算的求导法则,可以得到如下一些重要的基本初等函数的导函数
\begin{thx}
	{\bf 基本初等函数的求导公式II}
	\begin{enumerate}[(1)]
		\item $(\tan x)'=\sec^2x$
		\item $(\cot x)'=-\csc^2x$
		\item $(\sec x)'=\tan x\sec x$
		\item $(\csc x)'=-\cot x\csc x$
	\end{enumerate}
\end{thx}

证明略。

\bs
\egz 已知$f(x)$可导,且无零点,证明:$y=f(x)$和$y=f(x)\sin x$
在相交的位置必相切。

证:由乘法的求导法则
$$[f(x)\sin x]'=f'(x)\sin x+f(x)\cos x.$$
注意到$f(x)=f(x)\sin x$当且仅当$x=2k\pi+\frac{\pi}2$,
此时
$$[f(x)\sin x]'_{x=2k\pi+\frac{\pi}2}=f'(2k\pi+\frac{\pi}2),$$
这意味着两个函数的曲线在这些交点上的切线斜率相同,也即恰好是相切的。\fin

\begin{shaded}
	{\bf 关于$f(x)\sin x$的图像}
	
	在本门课程中,形如$f(x)\sin x$的函数常常用来作为讨论的示例,为此,了解一下
	有关函数的大致形态是很有必要的。请自行分析一下这些函数的图像有什么共性和差异。
	\begin{center}
		\resizebox{!}{4cm}{\includegraphics{./images/ch02/x3Sinx.pdf}}\quad
		\resizebox{!}{4cm}{\includegraphics{./images/ch02/xSinx.pdf}}
		
		$y=x^{1/3}\sin x$\hspace{5cm}$y=x\sin x$
		
		\resizebox{!}{4cm}{\includegraphics{./images/ch02/1xSinx.pdf}}\quad
		\resizebox{!}{4cm}{\includegraphics{./images/ch02/rxSinx.pdf}}
		
		$y=\df1x\sin x$\hspace{5cm}$y=[x]\sin x$
		
		\resizebox{!}{4cm}{\includegraphics{./images/ch02/lnxSinx.pdf}}\quad
		\resizebox{!}{4cm}{\includegraphics{./images/ch02/11xSinx.pdf}}
		
		$y=\ln x\sin x$\hspace{5cm}$y=1.1^x\sin x$
		
		\kaishu $y=f(x)$和$y=f(x)\sin x$的图像在相交的位置必相切
	\end{center}
\end{shaded}

\subsection{反函数求导法则}

自变量和因变量(函数)是一种相对的关系,将原来的因变量视为自变量,通过反函数来研究
原来的自变量的变化规律,是一种常见的数学手段。为此,讨论反函数的变化率(导数)就显得非常
有必要了。

假设函数$y=f(x)$可逆,其反函数为$x=g(y)$。设$\Delta x$和$\Delta y$
为自变量和因变量在$x$处对应的变化量。若$f(x)$可导,则必连续,从而
$$\lim\limits_{\Delta x\to 0}\Delta y=0,$$
又$f(x)$可逆,故必为一一映射,从而可知$\Delta x\to0\Leftrightarrow\Delta y\to0$。
于是由
$$f'(x)=\lim\limits_{\Delta x\to 0}\df{\Delta y}{\Delta x},$$
如果$f'(x)\ne 0$,则上式两边求倒数可得
$$
	\df1{f'(x)}=\lim\limits_{\Delta x\to 0}\df{\Delta x}{\Delta y}
	=\lim\limits_{\Delta y\to 0}\df{\Delta x}{\Delta y}
$$
再由导数的定义,上式右端就是$g'(y)$。

至此,我们可以得到如下的结论:

\begin{thx}
	{\bf 反函数求导法则:}设$y=f(x)$和$x=g(y)$互为反函数,若$f(x)$可导,且$f'(x)\ne 0$,则
	$$g'(y)=\df1{f'(x)}.$$
\end{thx}
注意,{\baa 使用反函数的求导公式时,必须注意原函数和反函数的自变量是不同的,
因此求导的时候$f'(x)$和$g'(y)$实际上分别是$f'_x(x)$和$g'_y(y)$。}
例如函数$y=e^x$,我们已知$(e^x)'_{\b x}=e^x$,故
$$(\ln y)'_{\b y}=\df1{(e^x)'_{\b x}}=\df1{e^x}=\df1y.$$

\bs
利用反函数的求导法则,我们可以验证了推导一些新的基本初等函数的求导公式:
\begin{thx}
	{\bf 基本初等函数的求导公式III}
	\begin{enumerate}[(1)]
		\item $(\arcsin x)'=\df1{\sqrt{1-x^2}}$
		\item $(\arccos x)'=-\df1{\sqrt{1-x^2}}$
		\item $(\arctan x)'=\df1{1+x^2}$
		\item $(\mathrm{arccot}x)'=-\df1{1+x^2}$
	\end{enumerate}
\end{thx}

证:(1)记$y=\sin x$,则
$$(\arcsin x)'_{\b x}=\df1{(\sin y)'_{\b y}}=\df1{\cos y}
=\df1{\sqrt{1-\sin y}}=\df1{\sqrt{1-x^2}}.$$

(2)注意到
$$\arccos x=\df{\pi}2-\arcsin x,$$
故
$$(\arccos x)'=\left(\df{\pi}2-\arcsin x\right)'
=-\df1{\sqrt{1-x^2}}.$$

(3)记$y=\tan x$,则
$$(\arctan x)'=\df1{(\tan y)'}=\df1{\sec^2y}
=\df1{1+\tan^2y}=\df1{1+x^2}.$$

(4)注意到
$$\mathrm{arccot x}=\df{\pi}2-\arctan x,$$
与(2)同理可证。
\fin

\bs
{\bf 课堂练习:} 设$f(x)=x^5+2x^3+4x$,$g(x)$
是$f(x)$的反函数,则$g'(7)=\underline{\quad\quad}$.

\ifhint
提示:$f(x)=7$时,$x=1$,故由反函数的求导公式
$$g'(7)=\df1{f'(1)}=\df1{15}.$$
\fi

\subsection{复合函数的求导法则}

利用函数的复合运算来构造新的函数,或者说将一些函数看成是常见函数的复合,
也是常见的一种情况,这时我们常常需要讨论复合函数的极限。

假设$u=f(x),y=g(u)$均可导,于是若将$y$视为$x$的复合函数,设
$\Delta x,\Delta u,\Delta y$为相对应的三个变化量,则有
\begin{align*}
	\lim\limits_{\Delta x\to 0}\df{\Delta y}{\Delta x}
	&=\lim\limits_{\Delta x\to 0}\df{\Delta y}{\Delta u}
	\cdot\df{\Delta u}{\Delta x}
	=\lim\limits_{\Delta x\to 0}\df{\Delta y}{\Delta u}
	\cdot\lim\limits_{\Delta x\to 0}\df{\Delta u}{\Delta x}\\
	&=\lim\limits_{\Delta u\to 0}\df{\Delta y}{\Delta u}
	\cdot\lim\limits_{\Delta x\to 0}\df{\Delta u}{\Delta x}
	=g'(u)f'(x)=g'(f(x))f'(x),
\end{align*}
于是,我们得到了如下的关于复合函数的求导法则
\pss{\baa 注意区分$[g(f(x))]'$和$g'(f(x))$,前者表示
	复合函数$g(f(x))$关于$x$求导,后者表示$g(u)$关于$u$求导,
	然后带入$u=f(x)$.}
\begin{thx}
	{\bf 复合函数求导法则:}设$f(x),g(x)$均可导,则
	$$[g(f(x))]'=g'(f(x))\cdot f'(x).$$
\end{thx}
试想一下,如果是三个可导的函数相互复合,则有
$$[h(g(f(x))]'=h'(g(f(x)))\cdot g'(f(x))\cdot f'(x),$$
这很类似于一环一环地解开一个相互连接的链条,因此该法则也被称为{\it 链式法则}。

\bs
利用链式法则,我们可以得到最后两个常用的初等函数求导公式
\egz 幂函数和指数函数的导函数
\ps{我们前面只给出了整数次幂的幂函数的求导公式}
\begin{thx}
	{\bf 初等函数的求导公式IV}
	\begin{enumerate}[(1)]
	  \item $(x^a)'=ax^{a-1}\;(a\ne 1)$
	  \item $(a^x)'=ax^{a-1}\;(a>0,a\ne 1)$
	\end{enumerate}
\end{thx}

证:(1)
$$(x^a)'=\left(e^{a\ln x}\right)'=e^{x\ln a}\df ax=ax^{a-1}.$$

(2)
$$(a^x)'=\left(e^{x\ln a}\right)'
=e^{x\ln a}\ln a=a^x\ln a$$
\fin

请注意,到目前为止,我们终于得到了所有基本初等函数的求导公式。接下来,我们可以利用
这些基本的公式,结合求导法则,来求各种函数的导函数了。

\bs
{\bf 课堂练习:} 计算下列函数的导函数
\begin{enumerate}[(1)]
  \setlength{\itemindent}{1cm}
  \item $y=e^{x^2}$ 
  \item $y=\sin (3x+2)$ 
  \item $y=\cos^2(1-2x)$ 
  \item $y=\ln\sin e^{-x}$ 
  \item $y=(1-30x)^{50}$ 
  \item $y=\ln(1+x^2)$ 
  \item $y=e^{\sqrt{1-3x}}$ 
  \item $y=x^x$ 
  \item $y=e^{\tan\frac 1x}$
\end{enumerate}

注:对形如$y=u^v$的函数求导,并没有任何基本的求导公式可用,因为
该函数不属于基本的初等函数,因此只能将其视为基本初等函数的复合函数
来考虑,也即
$$\left(u^v\right)'=\left(e^{v\ln u}\right)'
=e^{v\ln u}\left(v'\ln u+v\df1u\right)
=u^{v-1}\left(uv'\ln u+v\right).$$
例如:
$$(x^x)'=\left(e^{x\ln x}\right)'
=e^{x\ln x}\left(\ln x+1\right)=x^x(\ln x+1).$$
$$(x^{\sin x})'=\left(e^{\sin x\ln x}\right)'
=e^{\sin x\ln x}\left(\cos x\ln x+\df{\sin x}x\right)
=x^{\sin x-1}(x\cos x\ln x+\sin x).$$

对于这类题目,另一个常见的求导方法是所谓的“{\it 对数求导法}”,
先对函数取对数,然后再求导,例如:
$$\ln(u^v)=v\ln u,$$
两边分别求导
$$[\ln(u^v)]=\df{(u^v)'}{u^v},
\quad (v\ln u)'=v'\ln u+\df vu,$$
故
$$(u^v)'=u^v\left(v'\ln u+\df vu\right).$$
对于多个函数相乘构成的函数求导,采用先取对数再求导的方式也是非常高效的。

{\bf 思考:} 求下列函数的导函数
\begin{enumerate}[(1)]
  \setlength{\itemindent}{1cm}
  \item $y=x\sqrt{\df{1-x}{1+x}}$,
  \item $y=\df{x^2}{1-x}\sqrt[3]{\df{3-x}{(3+x)^2}}$. 
\end{enumerate}

\ifhint
提示:(1)
$$\ln y=\ln x+\df12\ln(1-x)+\df12(1+x),$$
于是
$$\df{y'}y=\df1x-\df1{2(1-x)}+\df1{2(1+x)},$$
故
$$y'=x\sqrt{\df{1-x}{1+x}}
\left[\df1x-\df1{2(1-x)}+\df1{2(1+x)}\right].$$

(2)略。
\fi

\begin{ext}
	{\centering\bf 习题2.2}
	
	\begin{enumerate} 
	  \item 计算如下函数的导函数
	  \begin{enumerate}[(1)]
	    \item $y=\arcsin\sqrt{1-x^2}$,
	    \item $y=\ln(e^x+\sqrt{1+e^2x})$,
	    \item $y=\arctan\sqrt{x^2-1}-\df{\ln x}{\sqrt{x^2-1}}$,
	    \item $y=\ln\tan\df x2-\cos x\ln\tan x$,
	    \item $y=\left(1+\df1x\right)^x$,
	    \item $y=x^2+2^x+x^x+2^2$,
	    \item $y=\sqrt{x+\sqrt{x+\sqrt x}}$.
	  \end{enumerate}
% 	  \item 已知函数$g(x)$在$x=a$连续,问函数$f(x)=|(x-a)|g(x)$
% 	  在$x=a$是否可导?若可导,证明之;若不可导,讨论增加什么样的条件可以使之可导。
% 	  利用以上讨论的结果,判断$f(x)=(x^2-4)|x^2+3x+2|$有几个不可导的点。
	  \item 设$f(x)$可导,且$f\,'\left(\df{\pi}{4}\right)=1$,求
		$$\left.f'\left(\arctan\df{1+x}{1-x}\right)\right|_{x=0}
		\quad\mbox{和}\quad  
		\left[f\left(\arctan\df{1+x}{1-x}\right)\right]'_{x=0}.$$
% 		在$x=0$处的导数。
% 	  \item 求$\left(1+\df1x\right)^x$和$x\sqrt{\sin x\sqrt{1-e^x}}$的导函数。
	  \item 自行完成不少于100道各种求导计算题({\it\b 不用写在作业本上})。 
	\end{enumerate}
\end{ext}

\section{高阶导数}

\subsection{高阶导数的定义}

高阶导数就是导函数的导数,并且可以依次递推,
\begin{thx}
	{\bf $f(x)$的$n$阶导数}:
	$$f^{\,(n)}(x)=\df{\d^nf(x)}{\d x^n}
	=\df{\d}{\d x}f^{(n-1)}(x)
	=\left[f^{\,(n-1)}(x)\right]'_x$$
\end{thx}

\egz 求函数$f(x)=x^3+2x^2-3x+10$的各阶导函数。

解:
\begin{align*}
	f'(x)&=3x^2+4x-3\\
	f''(x)&=6x+4\\
	f'''(x)&=6\\
	f^{(n)}(x)&=0,\;(n\geq 4)
\end{align*}
\fin

通过这道例题不难看出多项式函数在可导性方面
的一些“优良”性质:
\begin{itemize}
	\setlength{\itemindent}{1cm}
	\item 任意多项式函数都是任意阶可导的;
	\item 多项式函数的导函数也是多项式函数;
	\item 多项式函数的导数随着求导阶数的提高,次数不断降低;
	\item 若求导的阶数大于多项式的次数(最高次幂的幂次),则
	对应的高阶导数恒为零。
\end{itemize}

\bs
对于一般的函数,如果要求出其各阶导函数,
则需要通过观察寻找规律,通过归纳的方法给出其
高阶导数的一般表达式。很遗憾的是,大多数函数的高阶导数
是无法找到通用的表示的。下面是一些高阶导数有比较明显规律
的例子:

\egz 验证以下$n$阶导数公式
\begin{enumerate}[(1)]
	\setlength{\itemindent}{1cm}
	\item $\left(e^x\right)^{(n)}=e^x$
	\item $\left(x^a\right)^{(n)}=\left(ax^{a-1}\right)^{(n-1)}
	=\ldots=a(a-1)(a-2)\ldots(a-n+1)x^{a-n}$
	\item $\left(a^x\right)^{(n)}=a^x\left(\ln a\right)^n$
	\item $\left(\df1x\right) ^{(n)}=(-1)^n\df{n!}{x^{n+1}}$ 
	\item $\left(\ln(1+x)\right)^{(n)}=\left(\df1{1+x}\right)^{(n-1)}
	=\left(\df{-1}{(1+x)^2}\right)^{(n-2)}=\ldots
	=\df{(-1)^{n-1}(n-1)!}{(1+x)^{n}}$
	\item $\left(\ln(1-x)\right)^{(n)}=\df{-(n-1)!}{(1-x)^{n}}$
	\item $(\sin x)^{(n)}=\sin\left(x+\df{n\pi}2\right)$
	\item $(\cos x)^{(n)}=\cos\left(x+\df{n\pi}2\right)$  
	\item $(xe^x)^{(n)}=(n+x)e^x$
\end{enumerate}

\bs
\egz 求以下函数的高阶导数
\begin{enumerate}[(1)]
	\setlength{\itemindent}{1cm}
	\item $\ln\df{1+x}{1-x}$
	\item $\df{1+x}{1-x}$
	\item $\df{1+x^2}{1-x}$
\end{enumerate}

解:(1)\;因为$\ln\df{1+x}{1-x}=\ln(1+x)-\ln(1-x)$,故
\begin{align*}
	\left(\ln\df{1+x}{1-x}\right)^{(n)}
	=[\ln(1+x)]^{(n)}-[\ln(1-x)]^{(n)}
	=(n-1)!\left[\df{(-1)^{n-1}}{(1+x)^n}+\df1{(1-x)^n}\right].
\end{align*}

(2)\;因为$\df{1+x}{1-x}=-\df1{1-x}+2$,故
\begin{align*}
	\left(\df{1+x}{1-x}\right)^{(n)}
	=\left(-\df1{1-x}+2\right)^{(n)}=-\df{n!}{(1-x)^{n+1}}
\end{align*}

(3)\;因为$\df{1+x^2}{1-x}=-1-x+\df2{1-x}$
\ps{对任意形如$\df{P(x)}{x+a}$($P(x)$为多项式函数)的函数,
都可以用多项式的竖式除法将其分解为一个多项式加一个真分式
的形式,然后求其高阶导数},故
\begin{align*}
	\left(\df{1+x^2}{1-x}\right)^{(n)}
	=\left(-1-x+\df2{1-x}\right)^{(n)}
	=\left\{ \begin{array}{ll}
		-1+\df2{(1-x)^2},& n=1,\\
		\df{2(n+1)!}{(1-x)^{n+1}},& n\geq 2.
	\end{array} \right.
\end{align*}
\fin

\bs
{\bf 课堂练习:}求$\df{1+x+x^2}{1-x}$的$n$阶导函数。

\ifhint
提示:$\df{1+x+x^2}{1-x}=-x-2+\df3{1-x}$.
\fi

\bs
\egz 求$e^{ax}\sin bx$的$n$阶导数。

解:令$\phi=\arctan\frac ba$,则
\begin{align*}
	(e^{ax}\sin bx)'
	&=e^{ax}(a\sin bx+b\cos bx)
	=\sqrt{a^2+b^2}e^{ax}\sin\left(bx+\phi\right),\\
	(e^{ax}\sin bx)''
	&=\left[\sqrt{a^2+b^2}e^{ax}\sin(bx+\phi)\right]'\\
	&=\sqrt{a^2+b^2}e^{ax}(a\sin(bx+\phi)+b\cos(bx+\phi))\\
	&=(a^2+b^2)e^{ax}\sin(bx+2\phi),\\
	\ldots&\ldots\ldots\\
	(e^{ax}\sin bx)^{(n)}
	&=\left[(a^2+b^2)^{\frac{n-1}2}e^{ax}\sin(bx+(n-1)\phi)\right]'\\
	&=(a^2+b^2)^{\frac n2}e^{ax}\sin(bx+n\phi).
\end{align*}
\fin

注:对于形如$a\sin x+b\cos x$的公式的常用处理技巧:
令$\phi=\arctan\frac ba$,则
$$\df{a}{\sqrt{a^2+b^2}}=\cos\phi,
\quad\df{b}{\sqrt{a^2+b^2}}=\sin\phi$$
从而
$$a\sin x+b\cos x=\sqrt{a^2+b^2}\sin(x+\phi).$$

\subsection{Leibniz公式}

对于形如$u(x)v(x)$构造的函数,有如下的求高阶导数的公式:

\begin{thx}
	{\bf 求$n$阶导数的Leibniz公式:}设$u,v$均$n$阶可导,则
	$$\left[u(x)v(x)\right]^{(n)}=
	\sum\limits_{k=0}^nC_n^ku^{(n-k)}(x)v^{(k)}(x).$$
\end{thx}

用数学归纳法可以证明该定理,具体过程略。

利用Leibniz公式,可以比较方便地求一些特定的乘积形式的函数的
高阶导数。

\bs
\egz 设$y=x^3e^x$,求$y^{(10)}$。

解:记$u(x)=x^3,v(x)=e^x$,则
\begin{align*}
	& u'=3x^2, \; u''=6x, \; u'''=6, \; u^{(k)}\equiv 0\;(k\geq 4)\\
	& v^{(k)}\equiv e^x\;(k=1,2,3,\ldots)
\end{align*}
于是由Leibniz公式,
\begin{align*}
	(xe^x)^{(10)}
	&=uv^{(10)}+C_{10}^1u'v^{(9)}+C_{10}^2u''v^{(8)}+u'''v^{(7)}\\
	&=x^3e^x+10\cdot 3x^2e^x+45\cdot 6xe^x+120\cdot 6e^x\\
	&=(x^3+30x^2+270x+720)e^x
\end{align*}
\fin

类似这样的例子还有很多,比如:

$$(x^2\sin x)^{(80)}=(x^2-6320)\sin x-160x\cos x,$$

$$\left(\df{2x}{1-x^2}\right)^{(n)}=n!\left[\df1{(1-x)^{n+1}}
-\df{(-1)^n}{(1+x)^{n+1}}\right].$$

{\bf 课堂练习:}设$f(x)=(x^2+x+1)\sin^2 x$,求$f^{(10)}(0)$。

\ifhint
提示:$\sin^2x=\df12(1-\cos 2x)$,故
\begin{align*}
	[(x^2&+x+1)\sin^2x]^{(10)}	
	=\df12\left\{x^2+x+1-(x^2+x+1)\cos2x\right\}^{(10)}\\
	&=\df12\left[2^{10}(x^2+x+1)\cos 2x
	+10\cdot2^9(2x+1)\sin2x
	-90\cdot2^8\cos2x\right].
\end{align*}
带入$x=0$,化简可得
$$f^{(10)}(0)=-86\cdot2^7=-11008.$$
\fi

\bs
有时,也存在求特定点处的高阶导数的一些特殊方法和技巧,例如:

\egz 设$y=\arctan x$,求$y^{(n)}(0)$

解:
$$y'=\df1{1+x^2}\quad \Rightarrow\quad (1+x^2)y'=1,$$
利用Leibniz公式,右式两边同时求$n-1$阶导数,可得
$$(1+x^2)y^{(n)}(x)+2(n-1)xy^{(n-1)}(x)+(n-1)(n-2)y^{(n-2)}(x)=0,$$
令$x=0$,可得
$$y^{(n)}(0)+(n-1)(n-2)y^{(n-2)}(0)=0\quad
\Rightarrow\quad y^{(n)}(0)=-(n-1)(n-2)y^{(n-2)}(0)$$
注意到$y'(0)=1,y''(0)=0$,故
$$y^{(n)}(0)=\left\{\begin{array}{ll}
0,\quad& n=2k\\
(-1)^k(2k)!,\quad& n=2k+1
\end{array}\right.\;k=0,1,2,\ldots$$
\fin

\begin{shaded}
以上的例子还有一些不同的解法,概要如下:

解法一:利用三角函数的特定构造形式递推。
\begin{align*}
	y'&=\df1{1+x^2}=\cos^2y=\cos y\sin\left(y+\df{\pi}2\right),\\
	y''&=\cos^2y\cos\left(2y+\df{\pi}2\right)
	=\cos^2y\sin2\left(y+\df{\pi}2\right),\\
	y'''&=2\cos^3y\sin3\left(y+\df{\pi}2\right),
\end{align*}
进而可以推测并验证
$$y^{(n)}=(n-1)!\cos^ny\sin n\left(y+\df{\pi}2\right).$$
记$z=\arctan\df1x=\df{\pi}2-y$,于是
$$y^{(n)}=(n-1)!\df{\sin n(\pi-z)}{(1+x^2)^{n/2}}=
(n-1)!\df{\sin\left(n\arctan\frac1x\right)}{(1+x^2)^{n/2}}.$$

解法二:利用幂级数展开的方法。考虑函数$\arctan x$的Taylor展开式:
$$(\arctan x)'=\df1{1+x^2}=\sum\limits_{n=0}^{\infty}(-1)^nx^{2n},
\quad (|x|<1).$$ 
通过幂级数的逐项积分,可得对任意$|x|<1$,
\begin{align*}
	\arctan x&=\dint_0^x\sum\limits_{n=0}^{\infty}(-1)^nt^{2n}\d t
	=\sum\limits_{n=0}^{\infty}(-1)^n\dint_0^xt^{2n}\d t\\
	&=\sum\limits_{n=0}^{\infty}(-1)^n\df{x^{2n+1}}{2n+1},
\end{align*}
记$a_{2n}=0,a_{2n+1}=\df{(-1)^n}{2n+1}$,于是由Taylor公式的定义可知,
$$(\arctan x)^{(n)}|_{x=0}=a_n\cdot n!
=\left\{\begin{array}{ll}
	0,& n=2k;\\ (-1)^k(2k)!, & n=2k+1.
\end{array}\right.\;k=0,1,2,\ldots$$

和刚才这个例子类似,还有如下的例题:

\egz 设$y=(\arcsin x)^2$,求$y^{(n)}(0)$

提示:
$$y'=2\arcsin x\df1{\sqrt{1-x^2}}\quad
\Rightarrow\quad \sqrt{1-x^2}y'=2\arcsin x,$$
两边同时求导得
$$-\df1{\sqrt{1-x^2}}y'+\sqrt{1-x^2}y''=2\df1{\sqrt{1-x^2}}\quad
\Rightarrow\quad-xy'+(1-x^2)y''=2,$$
两边同时再求$n$次导数,然后令$x=0$,得
$$y^{(n+2)}(0)=n^2y^{(n)}(0).$$
易得$y'(0)=0,y''(0)=2$,从而
$$y^{(n)}(0)=\left\{\begin{array}{ll}
2[(n-2)!!]^2,& n\mbox{为偶数}\\
0,& n\mbox{为奇数}
\end{array}\right.$$

注:以上结果中出现了{\it 双阶乘},其定义如下:
$$(2n)!!=2n(2n-2)(2n-4)\ldots2,
\quad (2n+1)!!=(2n+1)(2n-1)\ldots3\cdot1.$$
\end{shaded}

\subsection{反函数的高阶导数}

\bs
关于高阶导数,有一个看起来很奇怪的结论:

\egz 设$y=f(x)$和$x=g(y)$互为反函数,二者均可导。证明:
$$g''(y)=-\df{f''(x)}{[f'(x)]^3}.$$
利用该结果,继续推导$g'''(y)$关于$x$的表达式。

解:由反函数求导公式$g'(y)=\df1{f'(x)}$,于是
\begin{align*}
	g''(y)&=\left[g'(y)\right]'_y=\left[\df1{f'(x)}\right]'_y
	=\left[\df1{f'(g(y))}\right]'_y\\
	&=-\df{1}{[f'(g(y))]^2}[f'(g(y))]'_y
	=-\df{1}{[f'(g(y))]^2}f''(g(y))g'(y)\\
	&=-\df{1}{[f'(x)]^2}f''(x)\df1{f'(x)}
	=-\df{f''(x)}{[f'(x)]^3}.
\end{align*}
\begin{shaded}
	{\bf\baa 一个重要的注记,千万看清楚了!}
	
	这里暂停一下,看一看是否还有更为简洁的方法。事实上,
	有一种非常“形式化”的推导方法,在求解此类问题中有着
	广泛的应用。例如,对本题而言,可以这样表示
	\begin{tcolorbox}[colframe=red!80!black]
		\begin{align*}
			g''(y)&=\df{\d^2x}{\d y^2}
			=\df{\d }{\d y}\left(\df{\d x}{\d y}\right)
			=\df{\d }{\d y}\left(\df1{f'(x)}\right)\\
			&=\df{\d }{\b\d x}\left(\df1{f'(x)}\right)\cdot\df{\b\d x}{\d y}
			=-\df{f''(x)}{[f'(x)]^2}\cdot\df1{f'(x)}
			=-\df{f''(x)}{[f'(x)]^3}.
		\end{align*}
	\end{tcolorbox}
	在上面的推导中,$\d x,\d y$都变成了某种可以被直接代换的“量”(学习了微分之后,
	我们知道,如果按照导数作为微商的定义,其上下的这两个部分可以分别视为$x,y$的微分,
	微分没有数量上的意义,但可以用来刻画无穷小量之间的比值),从而使得推导过程变得
	非常简单清晰。
	
	类似的推导方式如果用来推导反函数和复合函数的求导公式,结果也会惊人地简单:
	\begin{itemize}
	  \item 反函数求导公式:设$y=f(x)$和$x=g(y)$互为反函数,则
	  $$g'(y)=\df{\d x}{\d y}=\df1{\df{\d y}{\d x}}=\df1{f'(x)};$$
	  \item 链式法则:设$u=f(x),y=g(u)$,则
	  $$[g(f(x))]'=\df{\d y}{\d x}
	  =\df{\d y}{\d u}\cdot \df{\d u}{\d x}
	  =g'(u)f'(x)=g'(f(x))f'(x).$$
	\end{itemize}
	
	接下来,我们使用该方法继续推导本例中$g'''(y)$的表达式:
\end{shaded}
\begin{align*}
	g'''(y)
	&=\df{\d^3 x}{\d y^3}
	=\df{\d }{\d y}\left(\df{\d^2 x}{\d y^2}\right)\\
	&=\df{\d }{\d y}\left(-\df{f''(x)}{[f'(x)]^3}\right)
	=\df{\d }{\d x}\left(-\df{f''(x)}{[f'(x)]^3}\right)
	\cdot\df{\d x}{\d y}\\
	&=-\df{f'''(x)[f'(x)]^3-f''(x)3[f'(x)]^2f''(x)}{[f'(x)]^6}
	\cdot\df1{f'(x)}\\
	&=-\df{f'''(x)f'(x)-3[f''(x)]^2}{[f'(x)]^5}
\end{align*}
\fin

%\bs
%\egz 已知$y(t),x(t)$均关于$t$可导,且$x'(t)\ne 0$,
%求$\df{\d y}{\d x}$和$\df{\d^2y}{\d x^2}$。
%
%解:
%\begin{align*}
	%\df{\d y}{\d x}&=\df{\df{\d y}{\d t}}{\df{\d x}{\d t}}
	%=\df{y'_t}{x'_t}\\
	%\df{\d^2 y}{\d x^2}&=\df{\d}{\d x}\left(\df{\d y}{\d x}\right)
	%=\df{\df{\d\df{\d y}{\d x}}{\d t}}{\df{\d x}{\d t}}
	%=\df{\df{\d\df{y'_t}{x'_t}}{\d t}}{\df{\d x}{\d t}}\\
	%&=\df{y''_{tt}x'_t-y'_tx''_{tt}}{(x'_t)^2}\cdot\df{1}{x'_t}
	%=\df{y''_{tt}x'_t-y'_tx''_{tt}}{(x'_t)^3}
%\end{align*}
%\fin
%
%{\bf 思考:}你能否按照以上思路,推导出$\df{\d^3y}{\d x^3}$的表达式?
%
%\ifhint
%提示:
%$$\df{\d^3y}{\d x^3}
%=\df{y'''_{ttt}(x'_t)^2-3y''_{tt}x''_{tt}x'_t
%+3y'_t(x''_{tt})^2-y'_tx'''_{ttt}x'_t}{(x'_t)^5}.
%$$
%\fi

\bs
\begin{ext}
	{\centering\bf 习题2.3}
	
	\begin{enumerate}
	  \item 设$y=\ln\sqrt{\df{1-x}{1+x}}$,求$y''(0)$。
	  \item 已知$f(x)$二阶可导,设$y=\df{f(x)}{x}$,
	  求$\df{\d^2y}{\d x^2}$。
	  \item 已知$f(x)=\left\{\begin{array}{ll}
	  	\ln(1+2x),& x>0, \\ x^2+2x, & x\leq 0,
	  \end{array}\right.$
	  求$f''(x)$。
	  \item 求下列函数的$n$阶导函数
	  \begin{enumerate}[(1)]
	    \item $y=\sin^2x$;
	    \item $y=x\ln x$;
	    \item $y=\df{x^2}{1-x}$。
	  \end{enumerate}
	  \item (选作)已知$f(x)=x^2\ln(1+x)$,求$f^{(n)}(0)$。
	\end{enumerate}
\end{ext}

\section{隐函数与参数方程求导}

函数一种刻画运动的手段,一元函数主要用于表示一维(一个自变量)的运动,
其轨迹是平面山的一条曲线。然而,正如我们所知,形如$y=f(x)$的函数
无法表达平面上所有的曲线,甚至连刻画圆这样一个简单规则的曲线都有些
“力不从心”,不得不将其分成上下两个部分,用两个函数来表达。从这个意义
上说,形如$y=f(x)$的函数在表达平面曲线的能力上是存在一定缺陷的,或者
说,它的{\it 表达能力不够强}。

幸运的是,在数学上,早已出现了比$y=f(x)$表达能力更强的函数形式,
其中的代表就是本节要讨论的隐函数和参数方程。

\subsection{隐函数求导}

所谓{\it 隐函数},通常就是关于一个或多个变量的等式(或者叫方程),
具体来说,和一元函数对应的隐函数就是一个关于变量$x,y$的方程(从这个
意义上说,$y=f(x)$显然也可以看作一种特殊的隐函数方程)。例如单位圆的
方程$x^2+y^2=1$,从它可以推导出圆的上下两个部分对应的函数
$y=\pm\sqrt{1-x^2}$,但从形式上看,它显然比后者更为简洁,而且几何
意义也更加明确(圆是到原点距离相同的点的集合)。和后者一个很大的不同还
在于,隐函数方程中往往不必实现指定那个(些)变量是自变量,哪个(些)是
因变量(函数),这虽然看似不利于计算所谓的函数值,但实际上并不会构成
是指的问题,因为它已经很好地表达了变量之间的对应关系,同时还去除了函数
必须是“一对一”或“多对一”这种人为的约束。

那么对于用隐函数形式给出的函数,该如何求导呢?来看下面的例子

\egz 设$y=y(x)$是由方程
$$x^3+y^3=3xy$$
所确定的隐函数,满足$y(3/2)=3/2$,求其
在点$(3/2,3/2)$处的切线方程。

分析:这个方程表示的是著名的{\kaishu Descartes叶形线},
\pss{\centering
	\includegraphics[width=0.8\marginparwidth]
	{./Images/Ch02/800px-Frans_Hals_-_Portret_van_René_Descartes.jpg}\\
	\href{https://en.wikipedia.org/wiki/René_Descartes}{René Descartes(1596-1650)}\\[1em]
	"I think, therefore I am"\\[1em]
	"近代哲学之父"\\[1em]
	"近代科学的始祖"
}
形如下图:
\begin{figure}[h]
	\centering
	\includegraphics[width=0.4\textwidth]{./Images/Ch02/x3y33xy.jpg}
	\caption{Descartes叶形线}
	\label{fig:x3y33xy}
\end{figure}

如果用形如$y=f(x)$的函数表示这条曲线,显然会相当复杂。

设我们所要求切线的点所在的一段曲线可以表示为$y=y(x)$,带入到方程中可得
$$x^3+y^3(x)=3xy(x),$$
这个方程左右都是关于$x$的函数,于是可以考虑两边求导,然后利用求导后的
结果来推导出$y'(x)$的表达式。

解:将$y$视为$x$的函数,对已知方程两边关于$x$求导,可得
\ps{在隐函数方程求导的结果中,允许同时包含$x$和$y$}
$$3x^2+3y^2y'=3y+3xy'\quad\Rightarrow\quad
y'=\df{x^2-y}{x-y^2},$$
带入$(x,y)=(3/2,3/2)$可得$y'(3/2)=-1$,故所求切线方程为
$$y=-x+3.$$
\fin

总结一下,由隐函数方程$f(x,y)=0$解出变量$y$关于$x$的导数大致的过程如下:
首先假设$y$为$x$的函数$y(x)$,则原方程即为
$$f(x,y(x))=0;$$
接下来两边同时关于$x$求导,然后利用求导后的方程解出$y'_x$。

\bs
\egz 设$y=y(x)$是由方程$y^2=x^2-\cos y$所确定的隐函数,求$y''$。

\begin{figure}[h]
	\centering
	\includegraphics[width=0.4\textwidth]
	{./Images/ch02/y2x2-cosy.jpg}
	\caption{$y^2=x^2-\cos y$}
	\label{fig:y2x2-cosy}
\end{figure}

解:设$y=y(x)$是由已知方程所确定的函数,于是
$$y^2(x)=x^2-\cos y(x).$$
两边关于$x$求导,可得
$$2yy'=2x+y'\sin y\quad\Rightarrow\quad
y'=\df{2x}{2y-\sin y}.$$
该式两边再同时关于$x$求导,可得
$$y''=\df{2(2y-\sin y)-2x(y'-y'\cos y)}{(2y-\sin y)^2}.$$
带入$y'$的表达式,化简后可得
$$y''(x)=\df{8y^2-8y\sin y+2\sin^2y-2x^2+2x^2\cos y}
{(2y-\sin y)^3}.$$
\fin

\egz 设$x^2+xy+y^2=1$,求$y''$。

\pss{\centering
	\includegraphics[width=0.9\marginparwidth]
	{./Images/ch02/x2xyy2.pdf}\\
	$x^2+xy+y^2=1$
}

提示:
$$y'=-\df{2x+y}{x+2y},\quad y''=-\df6{(x+2y)^3},\quad
y'''=-\df{54x}{(x+2y)^5}$$

注:函数曲线与$x+2y=0$的交点处的切线与$y$轴平行。

\begin{shaded}
	{\bf 隐函数求导的一个几何应用}
	
	\egz 证明椭圆上任一点处的法线平分该点与两个焦点的连线所夹角。

	证:如图
	\begin{center}
		\resizebox{!}{5cm}{\includegraphics{./images/ch02/rollEc.pdf}}
	\end{center}
	只需证明上半椭圆上任一点有此性质即可。通过旋转椭圆,总可以是点$P(x,y)$处的切线为水平的,
	即$y'(x)=0$。
	此时,设椭圆的两个焦点分别为$(0,0)$和$(h,k)$,则由椭圆的几何性质,有
	$$\sqrt{x^2+y^2}+\sqrt{(x-h)^2+(y-k)^2}=L,$$
	其中$L$为某个确定的值。两端同时对$x$求导,并带入$y'(x)=0$,可得
	$$\df{x}{\sqrt{x^2+y^2}}=\df{h-x}{\sqrt{(x-h)^2+(y-k)^2}}.$$
	由图上不难看出该式也即$\sin\alpha=\sin\beta$,即证。
	\fin
\end{shaded}

\subsection{参数方程求导法则}

参数方程也是一种常见的表示几何对象(例如平面曲线)的方式,而且和
隐函数方程具有一样强的表示能力。平面曲线的参数方程通常形如:
$$
	\left\{\begin{array}{l}
		x=x(t),\\
		y=y(t),
	\end{array}\right.\;t\in [a,b].
$$

在此,我们假设$x(t),y(t)$均可导,来考虑一下$y$关于$x$的导数。
参考上一节中的“形式化”推导方法,可以发现
\begin{thx}
	$${\b y'(x)}=\df{\d y}{\d x}=\df{\d y}{\d t}\df{\d t}{\d x}
	=\df{\df{\d y}{\d t}}{\df{\d t}{\d x}}={\b \df{y'(t)}{x'(t)}}$$
\end{thx}
以上公式也可以写为
$$y'_x=\df{y'_t}{x'_t}.$$
给出某个点$(x,y)$对应的$t$,利用上式右端的公式就可以求出在该点处
$y$关于$x$的导数(或者说切线斜率)了——但请不要忘记一个必要的前提,
$x'(t)\ne 0$\ps{事实上,如果将$x$视为自变量,显然必须要求$x(t)$是可逆的,
也即$x'(t)\ne 0$}。

不妨再进一步,考虑$y$关于$x$的二阶导数。
\begin{thx}
	\begin{align*}
		{\b y''(x)}&=\df{\d^2y}{\d x^2}=\df{\d}{\d x}y'(x)
		=\df{\d}{\d x}\left(\df{y'(t)}{x'(t)}\right)
		=\df{\d}{\d t}\left(\df{y'(t)}{x'(t)}\right)\cdot\df{\d t}{\d x}\\
		&=\df{y''(t)x'(t)-y'(t)x''(t)}{[x'(t)]^2}\cdot\df1{x'(t)}
		={\b \df{y''(t)x'(t)-y'(t)x''(t)}{[x'(t)]^3}}
	\end{align*}
\end{thx}
该公式也可以写为
$$y''_{xx}=\df{y''_{xt}}{x'_t}=\df{y''_{tt}x'_t-y'_tx''_{tt}}
{(x'_t)^3}.$$

掌握这种推导方法,对于更好地推导计算各类求导公式是非常重要的。

\bs
\egz 已知$\left\{\begin{array}{l}x=t-\sin t\\
y=1-\cos t\end{array}\right.$,求$y''(x)$。

\begin{figure}[h]
	\centering
	\includegraphics[width=0.7\textwidth]
	{./Images/Ch02/sphereRoll.jpg}
	\caption{圆滚线}
	\label{fig:sphereRoll}
\end{figure}

解:
$$\df{\d x}{\d t}=1-\cos t,\quad \df{\d y}{\d t}=\sin t,$$
故
$$y'(x)=\df{\d y}{\d x}=\df{\df{\d y}{\d t}}{\df{\d x}{\d t}}
=\df{\sin t}{1-\cos t}.$$
于是
$$\df{\d y'(x)}{\d t}=\df{\cos t(1-\cos t)-\sin t\sin t}
{(1-\cos t)^2}=\df{-1}{1-\cos t}.$$
于是
$$y''(x)=\df{\d^2y}{\d x^2}=\df{\d y'(x)}{\d x}
=\df{\df{\d y'(x)}{\d t}}{\df{\d x}{\d t}}
=\df{-1}{(1-\cos t)^2}.$$
\fin

\bs
\egz 设$\left\{\begin{array}{l}x=f\,'(t)\\ y=tf\,'(t)-f(t)
\end{array}\right.$,求$\df{\d^2y}{\d x^2}$,其中$f\,''(x)$存在且不为零。

解:
$$x'_t=f''(t),\quad y'_t=tf''(t),$$
于是
$$y'_x=\df{y'_t}{x'_t}=t.$$
进而
$$y''_{xt}=1,$$
于是
$$y''_{xx}=\df{y''_{xt}}{x'_t}=\df1{f''(t)}.$$
\fin

\begin{shaded}
	\egz 证明:{\kaishu Archimedes螺线}
	$C_1:\rho=a\theta$与双曲螺线$C_2:\rho=a/\theta$在交点处
	相互垂直。

	提示:如图,两曲线有无穷多个交点。除最左侧的交点外,两曲线在交点处
	对应的极角值是不同的(相差$2\pi$的整数倍)。
	具体来说,在从左至右的第$k+1$个交点处,两条曲线对应的极角值分别为
	$$\theta_1^{(k)}=\sqrt{k^2\pi^2+1}+k\pi,
	\quad \theta_2^{(k)}=\sqrt{k^2\pi^2+1}-k\pi.
	\quad (k=0,1,2,\ldots).$$
	\begin{center}
		\includegraphics[width=0.45\textwidth]
		{./images/ch02/arCurve.pdf}
		
		{\kaishu$C_1:\rho=\theta$和$C_2:\rho=1/\theta$
		\quad($\theta>0$)
		}
	\end{center}
	又两曲线的参数方程分别为
	$$
	\left\{\begin{array}{l}
		x=a\theta\cos\theta\\
		y=a\theta\sin\theta
	\end{array}\right.
	\quad\quad\quad
	\left\{\begin{array}{l}
		x=\df{a}{\theta}\cos\theta\\
		y=\df{a}{\theta}\sin\theta
	\end{array}\right.
	$$
	利用参数方程的求导法则,可求出相应的曲线斜率分别为
	$$k_1(\theta)=\df{\sin\theta+\theta\cos\theta}{\cos\theta-\theta\sin\theta}
	\quad\quad\quad
	k_2(\theta)=-\df{\theta\cos\theta-\sin\theta}{\theta\sin\theta+\cos\theta}
	$$
	注意到$\theta_1^{(k)}\theta_2^{(k)}=1$,不难验证,在所有的焦点上,
	均有$k_1(\theta_1^{(k)})k_2(\theta_2^{(k)})=-1$,
	这意味着两条曲线在交点处相互垂直!
\end{shaded}

\begin{ext}
	{\centering\bf 习题2.4}
	
	\begin{enumerate}  
	  \item 对下列函数,求$y''(x)$
	  \begin{enumerate}[(1)]
	    \item $y=\tan(x+y)$;
	    \item $y=1+xe^y$。
% 	    \item $x=t(1-\sin t),\;y=t\cos t$。
	  \end{enumerate}
	  \item 求曲线$\left\{\begin{array}{l}
	  	x=\df{3t}{1+t^2},\\ y=\df{3t^2}{1+t^2}.
	  \end{array}\right.$在$(0,0)$和$\left(\df32,\df32\right)$
	  处的切线方程。
	  \item 已知$\left\{\begin{array}{l}
	  	x=e^t\cos t\\ y=e^t\sin t
	  \end{array}\right.$,求$\left.\df{\d y}{\d x}\right|_{t=\frac{\pi}2}$
	  和$\left.\df{\d^2 y}{\d x^2}\right|_{t=\frac{\pi}2}$。
	  \item 设$x(t),y(t)$均三阶可导,试给出$y'''_{xxx}$关于$t$的表达式。
	  \item (选作)设曲线的极坐标方程为$\rho=\rho(\theta)$,求其对应的直角指标
	  方程$y=y(x)$相关的导数$y'_x$和$y''_{xx}$关于$\theta$的表达式。
	\end{enumerate}
\end{ext}

\section{相关变化率}

{\it 变化率}可以定义为,一个变量随另一个变量变化过程中,相对于后者发生变化的速率,或者
二者的相关的变化量的比值。对于连续变化的函数而言,导数所刻画的正是函数值关于自变量
的变化率。常见的变化率的例子包括:

\begin{itemize}
  \setlength{\itemindent}{2em}
  \item {\it 曲线的斜率}:函数值关于自变量的变化率 
  \item {\it 速度}:位移关于时间的变化率 
  \item {\it 密度}:质量关于体积/面积/长度的变化率 
  \item {\it 电流强度}:电量关于时间的变化率 
  \item {\it 边际收益}:收益关于投入的变化率 
  \item {\it 出生/死亡率}:人口随时间的变化率
\end{itemize}

在现实问题中,许多变量之间都存在着千丝万缕的联系,从它们基本的数量关系出发,
往往能够进一步去研究它们的变化率(导数)之间的关系。

在学习掌握了导数的概念和计算方法之后,本节所讨论的相关变化率问题是对导数
的一个初步应用。

\bs
\pss{
	\centering
	\includegraphics[width=0.9\marginparwidth]{./images/ch02/glassCup.jpg}
}
\egz 有一深度$8$m,上底直径$8$m的圆锥形容器,
以$4$m$^3$/min的速率向其中注水,当容器中水深$5$m时,水面上升的速度是多少?

解:如图\ref{fig:coneFilling},
\begin{figure}[h]
	\centering
	\includegraphics[width=0.35\textwidth]
	{./Images/Ch02/coneFilling.pdf}
	\caption{\exNo 图}
	\label{fig:coneFilling}
\end{figure}
设$t$时刻水面的高度为$h(t)$,水面的半径为$r(t)$,且设
容器中的水量为$V(t)$。显然,
$$r(t)=\df12h(t),$$
进而
$$V(t)=\df13\pi r^2(t)h(t)=\df{\pi}{12}h^3(t).$$
上式两边同时关于$t$求导,可得
$$V'(t)=\df{\pi}4h^2(t)h'(t)\quad\Rightarrow\quad
h'(t)=\df{4V'(t)}{\pi h^2(t)}.$$
由已知,注水的速度恒定,故$V'(t)=4$。设所求时刻为$t_0$,
则$h(t_0)=5$,与$V'(t_0)=4$一起代入,可得
$$h'(t_0)=\df{4}{5\pi},$$
即为所求。\fin

\bs
\pss{
	\centering
	\includegraphics[width=0.9\marginparwidth]{./images/ch02/whaleWatching.jpg}\\
	whale watching
}
\egz 甲乙两船分别向南和向东航行。在初始时刻,甲船恰位于乙船北方
$40$km处,后来在某一时刻测得甲船向南航行了20km,此时速度为15km/h;
乙船向东航行了15km,此时速度为25km/h。问该时刻两船是在相互靠近还是远离,
二者的相对速度是多少?

解:如图\ref{fig:shipDist},
\begin{figure}[h]
	\centering
	\includegraphics[width=0.5\textwidth]
	{./Images/Ch02/shipDist.pdf}
	\caption{\exNo 图}
	\label{fig:shipDist}
\end{figure}
设$t$时刻甲乙两船到乙船初始位置的距离分别为
$x(t),y(t)$,则两船之间的距离
$$S(t)=\sqrt{x^2(t)+y^2(t)},$$
于是
$$S'(t)=\df{x(t)x'(t)+y(t)y'(t)}{\sqrt{x^2(t)+y^2(t)}}.$$
在$t_1$时刻,$x(t_1)=15$,$x'(t_1)=25$,$y(t_1)=20$,
$y'(t_1)=-15$,带入计算即得
$$S'(t_1)=3.$$
也即在所述时刻,两船的距离以$3$km/h增大。\fin

\bs
{\bf 课堂练习:}质点$P$沿抛物线$x=y^2(y>0)$移动。
$P$的横坐标$x$的变化速度为$5$cm/s。
当$x=9$cm时,点$P$到原点的距离的变化速率是多少?

\ifhint 
 提示:
 $$\df{\d}{\d t}\sqrt{x^2+y^2}=\left(\sqrt{x^2+x}\right)'_xx'_t
 =\df{5(2x+1)}{\sqrt{x^2+x}}$$
 代入$x=9$,可得其值为$\df{95}{6\sqrt10}$(cm/s)
 \fi

\bs
\egz 钟表的时针和分针长度分别为$a$(cm)和$b$(cm),求$12:20$分,两针端点分离的速率。

\pss{
	\centering
	\includegraphics[width=0.8\marginparwidth]{./images/ch02/bigBen.jpg}\\
	Big Ben
}

提示:如图\ref{fig:clockFace},
\begin{figure}[h]
	\centering
	\begin{subfigure}[t]{0.4\textwidth}
		\centering
		\includegraphics[width=0.8\textwidth]
		{./Images/Ch02/clockFace1.pdf}
		\caption{$t$时刻}
	\end{subfigure}
	\begin{subfigure}[t]{0.4\textwidth}
		\centering
		\includegraphics[width=0.8\textwidth]
		{./Images/Ch02/clockFace2.pdf}
		\caption{$12:20$}
	\end{subfigure}
	\caption{\exNo 图:不同时刻表针的位置及其夹角}
	\label{fig:clockFace}
\end{figure}
两针的夹角变化速率$\theta'_t=\df{\pi}{30}-\df{\pi}{360}
=\df{11}{360}\pi$(弧度/分钟)。12:20时,$\theta=\df23\pi-\df{\pi}{18}
=\df{11}{18}\pi$。利用余弦定理
$$s^2=a^2+b^2-2ab\cos\theta$$
答案$0.38$(cm/min)

% \egz 圆形广场中央立着高度为$h$的灯柱,灯$A$位于灯柱顶端。
% 广场上任一点$P$处的照明强度$I$与该点到灯的距离的平方成反比,与光线与灯柱的
% 夹角的余弦成正比。一个人从距离灯柱$x$m处以速度$v$m/s沿径向离开柱子,求
% 其脚部的光照强度关于时间的变化率。
% 
% 提示:由已知
% $$I=k\df{h}{(x^2+h^2)^{\frac32}}.$$
% 从而
% $$I'_t=I'_xx'_t=vI'_x=-\df{xvkh}{(x^2+h^2)^{\frac52}}$$

\bs
\egz 一个长度为$l$的杆,一段连接半径为$r$的转轮,一段位于$x$轴上。转轮的中心
位于原点,以每分钟$m$转的速度逆时针旋转,求:杆位于$x$轴上的一端的运动速率。

\pss{
	\centering
	\includegraphics[width=\marginparwidth]{./images/ch02/crankshaft.jpg}\\
	活塞与曲轴
}

提示:如图\ref{fig:rollbar},
\begin{figure}[h]
	\centering
	\includegraphics[width=0.6\textwidth]
	{./Images/Ch02/rollbar.pdf}
	\caption{\exNo 图}
	\label{fig:rollbar}
\end{figure}
$$\sin\theta=\df{|AQ|}r,\quad \sin\alpha=\df{|AQ|}{l}$$
从而$\sin\alpha=\df rl\sin\theta$,进而$\alpha=\arcsin\left(
\df rl\sin\theta\right)$。
$$x=r\cos\theta+l\cos\alpha=r\cos\theta+l\cos
\left[\arcsin\left(\df rl\sin\theta\right)\right]$$
$x'_t=x'_{\theta}\theta'_t=2m\pi x'_{\theta}$,其中
$$x'_{\theta}
% =-r\sin\theta-r^2\sin\theta\cos\theta
% \df1{\sqrt{l^2-r^2\sin^2\theta}}
=-r\sin\theta\left(1-\df{r\cos\theta}{\sqrt{l^2-r^2\sin^2\theta}}\right)$$

\bs
\egz 边长$s$的正方体冰块在空气中融化,已知冰块的融化速度(体积减少速度)
与其表面积成正比,比例系数为$k(k>0)$。假设冰块融化的过程中始终保持正方体形状。经过
一个小时,其体积减少了四分之一,求其完全融化需要的时间。

\pss{\centering
\includegraphics[width=\marginparwidth]
{./images/ch02/iceCubeMelting.jpg}}

提示:如图\ref{fig:iceMelting}
\begin{figure}[h]
	\centering
	\includegraphics[width=0.35\textwidth]
	{./Images/Ch02/iceMelting.pdf}
	\caption{\exNo 图}
	\label{fig:iceMelting}
\end{figure}
设冰块体积为$V(t)$,由已知其融化速度为
$$V'(t)=-k(6s^2).$$
又$V=s^3$,故
$$V'(t)=3s^2s'(t).$$
从而可得$s'(t)=-2k$,解得$s=-2kt+s(0)$。令$t=1$,可得$s(1)-s(0)=-2k$。

注意到$s$递减的速度与$t$无关,故冰块完全融化需要的时间
$$T=\df{s(0)}{2k}=\df{s(0)}{s(0)-s(1)}=\df1{1-\df{s(1)}{s(0)}}.$$
其中
$$\df{s(1)}{s(0)}=\left[\df{V(1)}{V(0)}\right]^{\frac13}=
\left(\df34\right)^{\frac13}\approx0.91(h)$$
代入前式计算可得$T\approx11(h)$

类似本节的示例都可以说是微积分的应用,对于这类应用问题,在解题时一般可以参考如下
的步骤:
\begin{thx}
	{\bf 求解应用题的一般步骤}
	\begin{enumerate}
% 	  \setlength{\itemindent}{1cm}
	  \item {{\it 画图:}}理解题意,画出示意图
	  \item {{\it 定义变量:}}给出相关变量的数学表示
	  \item {{\it 建立数量关系:}}根据已知,建立方程
	  \item {{\it 数学推导:}}例如方程两边求导(或者积分)
	  \item {{\it 求得结果:}}整理推导后的结果,得出解答
	\end{enumerate}
\end{thx}

{\bf 课堂练习:}设一个身高$1.8$m的男子向着远离灯柱的方向
前进。已知灯柱高度为$4.5$米,男子步行的速度是每秒$2$m,问
当男子距离灯柱$15$米时,其影子变长的速度。

\begin{figure}[h]
	\centering
	\includegraphics[width=0.6\textwidth]
	{./Images/Ch02/manShadow.pdf}
	\caption{课堂练习题示意图}
	\label{fig:manShadow}
\end{figure}

\begin{shaded}
	{\bf 万有引力定律的推导}
	\pss{\centering
	\includegraphics[width=0.7\marginparwidth]
	{./images/ch02/GodfreyKneller-IsaacNewton-1689.jpg}\\
	\href{https://en.wikipedia.org/wiki/Isaac_Newton}
	{Isaac Newton(1642-1726)}\\[1em]
	Alexander Pope: {\it Nature and nature's laws lay hid in night;
	God said "Let Newton be" and all was light.}\\[1em]
	自然和她的法则在黑夜隐藏,\\
	上帝说,让牛顿去吧\\
	于是一切都已照亮\\[1em]
	“天不生牛顿,万古如长夜。”
	}
	
	关于变化率和相关变化率最好的实例是Newton从Kepler
	三定律推导出万有引力定律的过程。
	下面的推导过程较长,能够看完并且理解,相信也是对
	导数这一章知识掌握情况一种很好的测试。

	在牛顿的工作之前,德国天文学家、物理学家和数学家Johannes Kepler(1571-1630)
	通过多年的天文观察,得到所谓的{\it 行星运动三大定律}
	——这三大定律使Kepler
	赢得了“天空立法者”的美名:
	\begin{enumerate}
		\setlength{\itemindent}{2em}
	  \item {\it 轨道定律:}所有行星分别都运行在椭圆轨道上,其对应的“太阳”位于
	  椭圆的一个焦点上
	  \item {\it 面积定律:}在同样的时间内,行星{\it 向径(极径)}(即由“太阳”位置
	  指向行星位置的向量)在轨道平面上扫过的面积相等
	  \item {\it 周期定律:}行星的公转周期的平方与它同太阳最远距离的立方成正比 
	\end{enumerate}
	\begin{center}
		\resizebox{!}{4cm}{\includegraphics{./images/ch02/Kepler.pdf}}
	\end{center}
	如图,绿色曲线为行星的椭圆轨道,极坐标系的原点$O$表示“太阳”的位置,点$P$为行星的位置。
	椭圆的极坐标方程为
	$$r=\df{p}{1-e\cos\theta},$$
	其中:
	\begin{itemize}
		\setlength{\itemindent}{2em}
	  \item {\it 焦参数}:$p=\df{b^2}a$
	  \item {\it 离心率}:$e=\sqrt{1-\df{b^2}{a^2}}$
	  \item $a,b$分别为椭圆的半长轴和半短轴长度
	\end{itemize}
	
	在$t$时刻,行星的位置可用其向径来表示
	\ps{在印刷体中,向量默认使用粗体字母表示,在手写体
	中,则使用在字母上方画箭头的记号来表示}:
	$$\bm{r}(t)=r(t)(\cos\theta,\sin\theta),$$
	其中:
	\begin{itemize}
		\setlength{\itemindent}{2em}
	  \item $r(t)=|\bm{r}(t)|$
	  为向径的长度({\it 模}),也即行星到太阳的{\it 距离}
	  \item $\theta=\theta(t)$如图所示,为当前位置对应的{\it 极角}
	  \item 记$\bm{r}_0=(\cos\theta,\sin\theta)$为$\bm{r}(t)$对应的
	  {\it 单位向量,也称为方向向量}
	\end{itemize}
	
	根据Newton第二定律,行星所受的力
	\ps{对向量求导等价于对其每个分量分别求导,
	例如:$\bm{r}=(x(t),y(t))$,则$\bm{r}'_t=(x'_t,y'_t)$}
	\begin{align}
		\bm{F}&=m\bm{a}=m\bm{r}''_{tt}\notag\\
		&=m\left(\df{\d^2(r\cos\theta)}{\d t^2},
		\df{\d^2(r\sin\theta)}{\d t^2}\right)\notag\\
		&=m((r''-r\omega^2)\cos\theta-(2r'\omega+r\omega')\sin\theta,
		(2r'\omega+r\omega')\cos\theta+(r''-r\omega^2)\sin\theta)\notag\\
		&=m(r''-r\omega^2)\bm{r}_0+m\left(\df{2r'\omega+r\omega'}
		{\omega}\right)\bm{r}'_0\tag{N.1}
	\end{align}
	其中:
	\begin{itemize}
		\setlength{\itemindent}{2em}
	  \item $m$为{\it 行星质量}
	  \item $\bm{a}$为行星的{\it 加速度向量}
	  \item {\it 角速度}:$\omega=\theta'_t$
	  \item {\it 径向速度(率)}:$r'=\df{\d r}{\d t}$
	  \item {\it 径向加速度(率)}:$r''=\df{\d^2 r}{\d t^2}$
	\end{itemize}
	
	记$\d A$为向径转过角度$\d\theta$所对应的椭圆面积,则
	$$\d A=\df12r^2\d\theta,$$
	由Kepler第二定律,单位时间内向径扫过的面积为常数(记为$c$),即
	$$c=\df{\d A}{\d t}=\df12r^2\omega.\eqno{(\mbox{N}.2)}$$
	
	记行星的公转周期为$T$,则经过$T$时间向径扫过的面积恰为整个椭圆的面积$\pi ab$,
	也即
	$$\pi ab=\dint_0^T\df{\d A}{\d t}\d t=cT=\df12r^2\omega T,$$
	由此可解出
	$$r^2\omega=\df{2\pi ab}T.$$
	
	式(N.2)两边求导,可得
	% \ps{事实上,若记行星的公转周期为$T$,容易解得$r^2\omega=\df{2\pi ab}T$}
	$$0=(r^2\omega)'=r(2r'\omega+r\omega'),$$
	由此可见(N.1)式中的第二部分恒为零,从而
	$$\bm{a}=\bm{r}''_{tt}=(r''-r\omega^2)\bm{r}_0,$$
	这表明{\it 行星在任一点处的加速度方向(也即受力方向)是沿着其向径方向,或者
	是与其和“太阳”连线相一致的}!
	
	下面回到椭圆方程,其可以改写成
	$$p=r(1-e\cos\theta),$$
	两边求二阶导数,可得
	\begin{align}
		0&=p''=r''-e(r^2\cos\theta)''\notag\\
		&=(r''-r\omega^2)(1-e\cos\theta)+r\omega^2\notag\\
		&=\df{r''-r\omega^2}rp+r\omega^2\notag
	\end{align}
	于是
	$$r''-r\omega^2=-\df{(r^2\omega)^2}{r^2p}=-4\pi^2\df{a^3}{T^2r^2}$$
	
	根据Kepler第三定律,$\df{a^3}{T^2}$为常数,记太阳的质量为$M$,则
	$$\bm{F}=m\bm{a}=m(r''-r\omega^2)\bm{r}_0
	=-\left(\df{4\pi^2a^3}{MT^2}\right)\df{Mm}{r^2}\bm{r}_0,$$
	记{\it 万有引力常数}
	$$G=\df{4\pi^2a^3}{MT^2}\approx 6.67\times 10^{-11}
	(\mbox{m}^3/\mbox{kg}\cdot\mbox{s}^2),
	$$
	也即
	$$\bm{F}=-G\df{Mm}{r^2}\bm{r}_0$$
	
	% \begin{shaded}
	
	万有引力定律说明了宇宙万物之间都存在相互的引力,
	且引力的作用方向在二者的连线上,
	其大小与二者的质量乘积成正比,与二者的距离平方成反比,
	比例系数为{\it 绝对常数}(
	即在宇宙范围内为常数,相对而言,前述推导中出现的
	常数如$c$,$a^3/T^2$都
	是和所处的“太阳-行星”系统相关的)。
	
	% \end{shaded}
	
	以上的论证及其结果,并不就是Newton个人所完整发现的,
	事实上也不是Newton最初的论证
	(发表在Newton的巨著《自然哲学的数学原理》一书上)。
	Kepler已经证明了,如果行星的轨迹是圆形,则符合万有引力定律。
	但如果轨道是椭圆形的,
	Kepler由于没能像Newton发明了微积分那样,拥有更合适的数学工具,
	因此得不出所要的结果。

	\begin{center}
		\includegraphics[width=0.6\textwidth]
		{./images/ch02/800px-NewtonsPrincipia.jpg}\\
		{\kaishu \href{https://en.wikipedia.org/wiki/Philosophiæ_Naturalis_Principia_Mathematica}{《自然科学的数学原理》,1687.5}}
	\end{center}
	
	另一方面,Newton最初的论证,只是说明了以上规律对我们
	所处的“太阳-行星”系统是正确的,
	是其后的科学家进一步证实了这一规律”放之四海而皆准“,
	适用于从一切天体运动到微观
	世界的广泛范围,而所有后续论证的基础是大量的实验观测。
	例如,万有引力常数$G$的值于
	1789年由Henry Cavendish(1731-1810)利用他所发明的
	{\it 扭称}得出的。
	
	基于万有引力定律,诞生了一些非常著名的天文学成果,例如:
	
	\begin{itemize}
		\setlength{\itemindent}{2em}
		\item 计算出哈雷彗星的轨道和运动周期
		\item 发现了海王星和冥王星
		\item 解释了潮汐的起因和规律
		\item 计算出第一、第二和第三宇宙速度,从而开启了人类航天活动的历史
	\end{itemize}
	
	利用万有引力定律,可以反向推导出Kepler的行星运动三定律。其中第二和第三定理
	的推导相对容易,下面给出由万有引力定律推导Kepler第一定律的大致过程。
	
	Newton第二定律的极坐标形式
	$$
	\left\{\begin{array}{l}
		F(r)=m(r''-r(\theta')^2)\\
		F(\theta)=m(r\theta''+2r'\theta')
	\end{array}\right.
	$$
	
	Binet公式:(参见Wikipedia关于\href{https://en.wikipedia.org/wiki/Binet_equation}{Binet Equation}的条目)
	$$F=-mh^2u^2\left(u''_{\theta\theta}+u\right)$$
	其中:$u=\df1r$,$h=\df Lm=r^2\theta'$,由角动量守恒定律,为常数。
	
	万有引力方程可以写为
	$$F=-mk^2u^2,$$
	其中$k^2=GM$(称为{\kaishu Guass常数},是一个与行星无关,而只与太阳有关的量)。综合以上
	两式,可得
	$$u''_{\theta\theta}+u=\df{k^2}{h^2},$$
	以下解方程。令$u=\xi+\df{k^2}{h^2}$,则方程化为
	$$\xi''_{\theta\theta}+\xi=0,$$
	这是一个波动方程,其解具有如下形式
	$$\xi=A\cos(\theta-\theta_0),$$
	进而
	$$r=\df1u=\df{h^2/k^2}{1+A[\cos(\theta-\theta_0)]h^2/k^2}$$
	这实质上就是椭圆的极坐标方程。

	关于Newton对于现代科学的贡献,可以参考
	\href{https://zh.wikipedia.org/wiki/艾萨克·牛顿}
	{Wikipedia中文版}上有关条目的篇首简介:

	{\kaishu
	艾萨克·牛顿爵士(Sir Isaac Newton,1643年1月4日-1727年3月31日):
	英格兰物理学家、数学家、天文学家、自然哲学家和煉金術士。
	1687年他发表《自然哲学的数学原理》,阐述了万有引力和三大运动定律,
	奠定了此后三个世纪里力学和天文学的基础,成为了现代工程学的基础。
	他通过论证开普勒行星运动定律与他的引力理论间的一致性,
	展示了地面物体与天体的运动都遵循着相同的自然定律;
	为太阳中心学说提供了强而有力的理论支持,并推动了科学革命。

	在力学上,牛顿阐明了动量和角动量守恒的原理。在光学上,
	他发明了反射望远镜,并基于对三棱镜将白光发散成可见光谱的观察,
	发展出了颜色理论。他还系统地表述了冷却定律,并研究了音速。

	在数学上,牛顿与戈特弗里德·莱布尼茨分享了发展出微积分学的荣誉。
	他也证明了广义二项式定理,提出了“牛顿法”以趋近函数的零点,
	并为幂级数的研究作出了贡献。

	在2005年,英国皇家学会进行了一场“谁是科学史上最有影响力的人”
	的民意调查,在被调查的皇家学会院士和网民投票中,
	牛顿被认为比阿尔伯特·爱因斯坦更具影响力。}
\end{shaded}

\begin{ext}
	{\centering\bf 习题2.5}
	
	\begin{enumerate}  
% 	  \item 圆形广场中央立着高度为$15$米的灯柱,有一盏灯位于灯柱顶端。
% 		广场上任一点处的照明强度$I$与该点到灯的距离的平方成反比,与光线与灯柱的
% 		夹角的余弦成正比。一个人从距离灯柱$10$米处以速度$1.5$米每秒沿径向离开柱子,求
% 		其脚部的光照强度关于时间的变化率。
	  \item 质点$P$沿抛物线$x=y^2(y>0)$移动。$P$的横坐标$x$的变化速度为$5$cm/s。
		当$x=9$cm时,点$P$到原点的距离的变化速率是多少?
	  \item 垂直向上发射一枚火箭,在其起飞点100km外设置一个观察站,在观察仰角
	  为$\pi/4$时,测得仰角的增加率为$0.1$弧度每秒,求此时火车的上升速率。
	  \item 长度为$6$米的梯子靠在墙角,梯子底部距离墙角$5$米,某一时刻梯子底部开始
	  向远离墙角的方向滑动,滑动的速度为$0.2$米每秒,问
	  \begin{enumerate}[(1)]
	    \item 此时梯子顶部下滑的速度是多少?
	    \item 由梯子、墙面和地面构成的三角形的面积随时间的变化率是多少?
	    \item 梯子和地面的夹角的以怎样的速率变化?
	  \end{enumerate}
	  \item 半径为$a$的圆球渐渐沉入盛有水的半径为$b(b>a)$的圆柱形容器中,若
	  球的下降速度恒为$c$,求球浸没入水中恰好一半时,容器内水面上升的速率。
	  (提示:球冠的体积$V=\df{\pi}3(3R-H)H^2$,其中$H$为球冠的高度)
% 	  \item (选作)设曲线的极坐标方程为$\rho=\rho(\theta)$,求其对应的直角指标
% 	  方程$y=y(x)$相关的导数$y'_x$和$y''_{xx}$关于$\theta$的表达式。
	\end{enumerate}
\end{ext}

\section{微分}

计算函数的值从来都是一个简单问题,这一点大多数刚
学习微积分不就的人都不会意识到。
例如一个段子:你被判了无期,假释的唯一可能是算出$\sin 1$的近似值,要求精确到小数点后
100位,而你的工具只有无限量使用的笔和纸。
就你现在的知识,你觉得可能在有生之年
完成吗?或者改换一下任务,算算$\ln 3$?

数学家们显然也意识到了这个问题,在想出了用导数刻画变化率的办法之后,
他们很快发现了它与{\it 近似计算}之间的联系。
请注意,这个故事很长,本节只是开篇,
等到下一章学习了Taylor公式后才能真正告一段落。

\subsection{局部线性化和“以直代曲”}

对于任意给定的函数,如果已知它在某点的函数值,能否给出它附近其他点处的函数值,
并且尽可能精确\ps{在现实里,绝对的精确是不存在的,数学上中的计算其实也是一样}?

一个自然的想法是,用一条尽可能简单,而且易于计算函数值的曲线来近似给定的曲线,
由于没有比{\it 直线}(或者叫"{\it 线性函数"})更简单的函数形式,因此它就成了第一个
尝试的对象,对应的想法被称为“{\it 以直代曲}”\ps{后面我们还将学习它的“高级版本”——“以曲代曲”},
数学上这样定义:
\begin{thx}
	设已知函数$f(x)$在$x_0$附近有定义,记$\Delta y=f(x_0+\Delta x)-f(x_0)$,
	如果存在某个与$\Delta x$无关的常数,使得
	$$\Delta y=A\cdot\Delta x+\circ(\Delta x),\;(\Delta x\to 0),$$
	则称$f(x)$可以被以直代曲,或者更规范的说法是{\bf $f(x)$在$x_0$处可微},还有
	一种说法是$f(x)$在$x_0$附近可以“{\bf 局部线性化}”。此时,我们称
	$$\left.\d y\right|_{x=x_0}=A\cdot\Delta x$$
	为{\bf $f(x)$在$x_0$处的微分},当$x$是自变量时,一般认为$\d x=\Delta x$,故
	上式一般就记为:
	$$\left.\d y\right|_{x=x_0}=A\cdot\d x.$$
\end{thx}
	
\begin{figure}[h]
	\centering
	\includegraphics[width=0.5\textwidth]{./Images/Ch02/dyDy.pdf}
	\caption{$\Delta y$和$\d y$的关系}
	\label{fig:dyDy}
\end{figure}
	
直观地说,$f(x)$在$x_0$可微,意味着在$x_0$附近,$f(x)$可以近似地表示为一个线性函数: 
$$f(x)\approx f(x_0)+f\,'(x_0)(x-x_0).$$
为此,我们通常称函数$y=f(x_0)+f\,'(x_0)(x-x_0)$的{\bf 局部线性化函数}。
下面我们将说明,这个所谓的局部线性化函数对应的恰好就是函数图像
在该点的切线。这意味着,可导和可微,对于我们目前讨论的函数而言,
事实上是等价的。

\pss{这个式子也常常写为
	$$\d y=\df{\d y}{\d x}\d x,$$
	由此很容易发现为什么导数也被称为微商了}
\begin{thx}
	{\bf 可导与可微的关系:}$f(x)$在$x_0$可微与$f(x)$
	在$x_0$可导等价,且
	$$\d y =f'(x_0)\d x.$$
\end{thx}

证:设$f(x)$在$x_0$可微,则存在$A$,使得
$$f(x)-f(x_0)=A(x-x_0)+\circ(x-x_0),\;(x\to x_0),$$
该式进行恒等变形可得
$$\df{f(x)-f(x_0)}{x-x_0}=A+\circ(1),\;(x\to x_0),$$
也即
$$A=\lim\limits_{x\to x_0}\df{f(x)-f(x_0)}{x-x_0},$$
故$f(x)$在$x_0$可导,且$f'(x)=A$。

另一方面,设$f(x)$在$x_0$可导,则有
$$f'(x_0)=\lim\limits_{x\to x_0}\df{f(x)-f(x_0)}{x-x_0},$$
从而易得
$$\lim\limits_{x\to x_0}\df{f(x)-f(x_0)
-f'(x_0)(x-x_0)}{x-x_0}=0,$$
故
$$f(x)-f(x_0)-f'(x_0)(x-x_0)=\circ(x-x_0),\;(x\to x_0),$$
也即
$$f(x)-f(x_0)=f'(x_0)(x-x_0)+\circ(x-x_0),\;(x\to x_0),$$
由此可知$f(x)$在$x_0$可微,且
$$\d y =f'(x_0)\d x.$$
\fin

\bs
\egz 设$f(x)=x^3+2x^2-3x+6$,求$\d f(x)$和$\d f(x)|_{x=1}$,
并求其在$(1,6)$处的局部线性化函数$L(x)$。

解:
$$f'(1)=\left.(3x^2+4x-3)\right|_{x=1}=4,$$
故
\ps{\baa 注意:$x$和$\d x$是相互独立的,因此不可以将
$x$的取值带入到$\d x$中。事实上,$\d x$应该总是被视为
一个不可分的整体。}
\begin{align*}
	\d f(x)& =(3x^2+4x-3)\d x,\\
	\d f(x)|_{x=1}&=4\d x.
\end{align*}
$f(x)$在$(1,6)$处的局部线性化函数为
$$y-6=4(x-1),$$
也即
$$y=4x+2.$$
\fin

\subsection{微分的运算法则}

\begin{thx}
	{\bf 微分与四则运算:}设$u(x),v(x)$可微(导),则
	\begin{enumerate}[(1)]
	  \item $\d (u\pm v)=\d u\pm \d v$
	  \item $\d(uv)=v\d u+u\d v$
	  \item $\d\df uv=\df{v\d u-u\d v}{v^2}$
	\end{enumerate}
	
	\bs
	{\bf 复合函数的微分:}设$y=g(u),u=f(x)$均可微,
	则$y=g[f(x)]$可微,
	$$\d y=g'(u)\d u=g'(u)f'(x)\d x$$		
\end{thx}

了解了导数和微分之间的密切联系,就可以很容易地由各种求导法则
推出以上的微分运算法则。例如:
$$(uv)'=u'v+uv',$$
也即
$$\df{\d(uv)}{\d x}=v\df{\d u}{\d x}+u\df{\d v}{\d x}.$$
由此易得
$$\d(uv)=v\d u+u\d v.$$

从形式上看,复合函数的微分运算规则更有链式法则的“环环相扣”的味道,
事实上,对于一阶的微分运算(与一阶导数对应),我们不必关心每个
变量究竟是因变量还是自变量,或者是中间变量,从结果上看都具有类似
$$\d y=y'_x\d x$$
的形式,比如在上面的结论中
$$\d y=f'(u)f'(x)\d x,$$
其中的$f'(u)g'(x)$就是$y$关于$x$的导数$y'_x$。
这种性质非常有利于我们计算各种(一阶)微分,被称为{\it 一阶微分的
形式不变性。}\ps{高阶微分不具有此性质}

\bs
\egz 求函数$y=e^{2x-1}\sin x$的微分。

解:
\begin{align*}
	\d y
	&=\d(e^{2x-1}\sin x)
	=e^{2x-1}\d\sin x+\sin x\d e^{2x-1}\\
	&=e^{2x-1}\cos x\d x+\sin x 2e^{2x-1}\d x
	=(\cos x+2\sin x)e^{2x-1}\d x.
\end{align*}
\fin

\subsection{微分的应用}

有了微分这种对函数的局部近似,我们就可以对一些已知函数值附近的函数值进行
有效的逼近,或者,对于可能产生的计算误差进行分析。

\bs
\egz 
有一批半径为$1$cm的球,为了提高球面的光洁度,准备镀上一层铜,
厚度为$0.01$cm,试估计每只球需要的用铜量(铜的密度为$8.9$g/cm$^3$)。

提示:所需的铜量
$$M=8.9\cdot\Delta V,$$
其中$\Delta V$为镀铜前后球的体积差,利用微分进行近似,
$$\Delta V\approx \d V=\d\left(\df43\pi R^3\right)
=4\pi R^2\d R=4\pi R^2\Delta R,$$
将$R=1,\Delta R=0.01$带入计算,可得
$$\Delta V\approx 0.13(\mbox{cm}^3),$$
最终每个球所需铜量约为$1.16$g。

同样是这个例子,我们换一个角度来提问:

\egz 如果上述的球体在加工时,半径最大允许$0.01$cm的误差,
估计一下球的体积的绝对误差和相对误差分别有多大。

提示:利用微分进行估计
$$\Delta V\approx \d V=\d\left(\df43\pi R^3\right)
=4\pi R^2\d R=4\pi R^2\Delta R,$$
将$R=1,\Delta R=0.01$带入计算,可得
$$\Delta V\approx 0.13(\mbox{cm}^3),$$
注意到$\Delta$关于$\Delta R$是严格单调的,故以上就是所求的
体积的{\it 绝对误差}的近似值。进而相对误差
$$e_r=\df{\Delta V}{V}\approx 3\df{\Delta R}{R}=3\%.$$

\bs
在现实中,要制造精确的曲线,事实上是不可能的。所以,{\it 在
允许的范围内}(误差可以接受),使用多个直线段来“构造”出曲线,
是非常常见的情况。例如,古代的桥梁设计中就不乏这样精彩的作品,
图\ref{fig:zhaozhouqiao}是著名的赵州桥,靠近观察,
不难发现,它所使用的各种石构件
的边缘都是由直线构成的。

\begin{figure}[h]
	\centering
	\includegraphics[width=0.75\textwidth]{./Images/Ch02/archBridge-1.jpg}\\
	\includegraphics[width=0.75\textwidth]{./Images/Ch02/zhaozhouqiao.jpg}
	\caption{赵州桥}
	\label{fig:zhaozhouqiao}
\end{figure}

\begin{ext}
	{\centering\bf 习题2.6}
	
	\begin{enumerate}  
	  \item 设$y=x^3-2x$,
	  \begin{enumerate}[(1)]
	    \item 计算在$x=2$处当$\Delta x$分别为$1,\;0.1,\;
	  	0.01$时的$\Delta y$和$\d y$;
	    \item 写出$x=2$时$y$的微分表达式。
	  \end{enumerate}
	  \item 已知$2^{xy}=x-y$,求$\d y|_{x=0}$。
	  \item 同济教材习题2-5:4
	  \item 设
	  $f(x)=\lim\limits_{t\to\infty}t^2
	  \left[e^{x+\frac2t}-e^x\right]\sin\df xt$,求$\d f(x)$。
	  \item 已知当$h\to 0$时,
	  $$f(x+2h)-f(x)=h\sqrt{x^2+2x}+\circ(h),$$
	  求$\d\left[f\left(\df1x\right)\right]$。
	  \item (选作)设$a>0$,$|x|<<a$(表示$|x|$远远小于$a$),证明近似公式
	  $$\sqrt[n]{a^n+x}\approx a+\df{x}{na^{n-1}}.$$
	\end{enumerate}
\end{ext}

\section{小结}

导数即函数的{\it 变化率}\ps{变化率的另一种定义是单位自变量对应的函数改变量,例如:
单位水平位移对应的垂直位移——即斜率(或坡度),单位长度对应的质量——即线密度},
或者说是函数值的改变量关于自变量的改变量的比率,
代数上将其定义为自变量为特定取值时以上比率的极限。现实世界里的速度、
斜率和密度等都可以用导数来加以描述。

本章的重点是{\it 导数的计算},其中涉及众多{\it 基本初等函数的求导公式}和{\it 四则运算、
复合运算、函数逆运算的求导法则},这些都必须熟练掌握,作到{\it “倒背如流”}。
高阶导数的计算是本章的一个难点,正确地进行归纳和使用Leibniz{\it 公式}常常
是解题的关键。

微分的概念比较抽象,但就一元函数而言,可导和可微是等价的,因此可以简单
地将微分理解为求导的另一种形式——虽然两者有着不同的解释和物理、几何意义。
微分的计算公式都有等价的求导公式。

本章中有专门的一节讨论与导数相关的应用问题——{\it 变化率与相关变化率},
这些问题看似平常,却常常称为学习掌握的难点,正确解题的关键在于
如何正确用数学的符号和语言去描述实际问题,用更专业的话来说,
就是如何做好数学建模。

\newpage

\ifanswer
\section*{作业参考解答}
% \addcontentsline{toc}{section}{作业参考解答}

\begin{center}
	\bf 2.1 导数的概念
\end{center}

\bigskip

1.已知$f(x)=\left\{\begin{array}{ll}
2e^x+b,& x\leq0\\ ax+\sin x,& x> 0
\end{array}\right.$
试确定$a,b$的值,使得$f(x)$在$x=0$处可导。

解:$f(x)$在$x=0$处可导,故必连续,从而
$$f(0-0)=2+b=f(0+0)=0,\quad\Rightarrow \quad b=-2.$$
又
$$f'_-(0)=(2e^x+b)'|_{x=0}=2.$$
$$f'_+(0)=\limx{0^+}\df{ax+\sin x-(2+b)}{x}
=a+\limx{0^+}\df{\sin x}x=a+1.$$
故要使$f(x)$在$x=0$处可导,必有$2=a+1$,从而$a=1$。
\fin

\bigskip

2.证明:曲线$xy=a^2$上任一点处的切线与两坐标轴构成的三角形面积不变。

解:在已知曲线上任一点$(x_0,y_0)$处,切线斜率为
$$k=\left(\df{a^2}{x}\right)'_{x=x_0}=-\df{a^2}{x_0^2}.$$
于是过该点的切线为
$$y=y_0-\df{a^2}{x_0^2}(x-x_0).$$
其在$x$轴和$y$轴上的截距分别为(注意:$y_0=\df{a^2}{x_0}$)
$$X=\df{y_0x_0^2}{a^2}+x_0=2x_0,\quad
Y=y_0+\df{a^2}{x_0^2}x_0=2y_0.$$
故该切线与两坐标轴所围三角形面积为
$$S=\df12|XY|=2|x_0y_0|=2a^2.$$
由$(x_0,y_0)$的任意性,即证。\fin

\bigskip

3.讨论函数
$y=\left\{\begin{array}{ll}
	x^2\sin\df1x,& x\ne0;\\ 0, & x=0.
\end{array}\right.$
在$x=0$处的连续性、可导性以及导函数的连续性。

解:当$x\ne 0$时,
$$y'(x)=2x\sin\df1x-\sin\df1x.$$
当$x=0$时,
$$y'(0)=\limx0\df{x^2\sin\df1x-0}{x-0}=\limx0x\sin\df1x=0.$$
由此可知$y(x)$在$x=0$处可导,且连续。

又因为
$$\limx02x\sin\df1x=0,\quad \limx0\sin\df1x\mbox{不存在},$$
故$\limx0y'(x)$不存在。由此可知$y(x)$的导函数在$x=0$处不连续。\fin

\bigskip

4.设对任意$x\in\mathbb{R}$,均有$f(x+2)=f(x)$,已知$f'(0)=1$,
证明$f(x)$在$x=2$可导,并求$f'(2)$。

解:因为$f(x+2)=f(x)$,故
$$\lim\limits_{\Delta x\to0}\df{f(2+\Delta x)-f(2)}{\Delta x}
=\lim\limits_{\Delta x\to0}\df{f(\Delta x)-f(0)}{\Delta x}
=f'(0)=1,$$
由此即知$f(x)$在$x=2$可导,且$f'(2)=1$。
\fin

\bigskip

5.已知曲线$y=f(x)$和曲线$y=\sin x$在原点相切(即二者的切线相同),
求$\limx0\df{f(3x)}x$。

解:曲线$y=f(x)$和曲线$y=\sin x$在原点相切,故
$$f(0)=\sin 0=0,\quad f'(0)=(\sin x)'|_{x=0}=1.$$
于是
$$\limx0\df{f(3x)}x=3\limx0\df{f(3x)}{3x}=3f'(0)=3.$$
\fin

\bigskip

6.已知$f'(a)f(a)\ne 0$,求
$\limx0\left[\df{f(a+x)}{f(a)}\right]^{\frac1{\sin x}}.$

解:
\begin{align*}
	\mbox{原式}&=\limx0\left[1+\df{f(a+x)-f(a)}{f(a)}\right]^{\frac1{\sin x}}\\
	&=\limx0\left[1+\df{f(a+x)-f(a)}{f(a)}\right]^{\frac{f(a)}{f(a+x)-f(a)}
	\frac{f(a+x)-f(a)}{f(a)}\frac1{\sin x}}\\
	&=\left\{\limx0\left[1+\df{f(a+x)-f(a)}{f(a)}\right]^{\frac{f(a)}{f(a+x)-f(a)}}\right\}
	^{\limx0\frac{f(a+x)-f(a)}x\frac{x}{\sin x}\frac1{f(a)}}\\
	&=e^{\frac{f'(a)}{f(a)}}
\end{align*}
\fin

\bigskip

7.已知函数$g(x)$在$x=a$连续,问函数$f(x)=|(x-a)|g(x)$
在$x=a$是否可导?若可导,证明之;若不可导,讨论增加什么样的条件可以使之可导。
利用以上讨论的结果,判断$f(x)=(x^2-4)|x^2+3x+2|$有几个不可导的点。

解:
$$\lim\limits_{\Delta x\to0}\df{f(a+\Delta x)-f(a)}{\Delta x}
=\lim\limits_{\Delta x\to0}\df{|\Delta x|g(a+\Delta x)}{\Delta x}
=\lim\limits_{\Delta x\to0}\df{|\Delta x|}{\Delta x}g(a+\Delta x),$$
该极限存在当且仅当$\lim\limits_{\Delta x\to0}g(a+\Delta x)=0$,也即
$\limx{a}g(x)=0$。

函数$f(x)=(x^2-4)|x^2+3x+2|$也即
$$f(x)=|(x+2)(x+1)|(x+2)(x-2),$$
根据前述的结论,在$x=-2$处$f(x)$可导,在$x=-1$处$f(x)$不可导。\fin

\bigskip

8.设对任意$x,y\in\mathbb{R}$,有
$$f(x+y)=f(x)+f(y)+x^2y+xy^2,$$
且当$x\to0$时$f(x)$与$x$是等价无穷小,证明$f(x)$处处可导,并求其导函数。

解:令$x=y=0$,由已知等式可得$f(0)=0$。

对任意$x_0\in\mathbb{R}$,由已知等式及$\limx{0}\df{f(x)}x=1$,
\begin{align*}
	\lim\limits_{\Delta x\to0}\df{f(x_0+\Delta x)-f(x_0)}{\Delta x}
	&=\lim\limits_{\Delta x\to0}\df{f(\Delta x)+x_0^2\Delta x+x_0\Delta x^2}
	{\Delta x}\\
	&=\lim\limits_{\Delta x\to0}\df{f(\Delta x)}{\Delta x}+x_0^2
	=1+x_0^2.
\end{align*}
由此可知$f(x)$在$x_0$处可导。因为$x_0$是任意的,故$f(x)$处处可导,其导函数为
$f'(x)=1+x^2$。\fin

\begin{center}
	\bf 2.2 函数的求导法则
\end{center}

\bigskip

\bigskip

3.设$f(x)$可导,且$f'\left(\df{\pi}{4}\right)=1$,求
$$\left.f'\left(\arctan\df{1+x}{1-x}\right)\right|_{x=0}
\quad\mbox{和}\quad  
\left[f\left(\arctan\df{1+x}{1-x}\right)\right]'_{x=0}.$$

解:
$$\left.f'\left(\arctan\df{1+x}{1-x}\right)\right|_{x=0}
=f'(\arctan 1)=f'\left(\df{\pi}{4}\right)=1.$$

\begin{align*}
	\left[f\left(\arctan\df{1+x}{1-x}\right)\right]'_{x=0}
	&=\left.f'\left(\arctan\df{1+x}{1-x}\right)
	\df{1}{1+\left(\frac{1+x}{1-x}\right)^2}
	\df{(1-x)+(1+x)}{(1-x)^2}
	\right|_{x=0}\\
	&=1\cdot\df12\cdot2=1.
\end{align*}
\fin

\begin{center}
	\bf 2.3 高阶导数
\end{center}

\bigskip

1.设$y=\ln\sqrt{\df{1-x}{1+x}}$,求$y''(0)$。

解:$y=\df12[\ln(1-x)-\ln(1+x)]$,
$$y'=\df12\left(-\df1{1-x}-\df1{1+x}\right)
=-\df1{1-x^2},$$
$$y''=-\df{-2x}{(1-x^2)^2}=-\df{2x}{(1-x^2)^2}.$$
故$y''(0)=0$.\fin

\bigskip

2.已知$f(x)$二阶可导,设$y=\df{f(x)}{x}$,求$\df{\d^2y}{\d x^2}$。

解:
$$y'=\df{f'(x)x-f(x)}{x^2},$$
$$y''=\df{f''(x)x^3-2x[f'(x)x-f(x)]}{x^4}=\df{f''(x)x^2-2f'(x)x+2f(x)}{x^3},$$
即为所求。\fin

\bigskip

3.已知$f(x)=\left\{\begin{array}{ll}
  	\ln(1+2x),& x>0, \\ x^2+2x, & x\leq 0,
  \end{array}\right.$
求$f''(x)$。

解:$x>0$时,$f'(x)=\df2{1+2x}$;$x<0$时,$f'(x)=2x+2$,
又
$$f'_-(0)=\lim\limits_{\Delta x\to0^-}\df{\Delta x^2+2\Delta x-0}{\Delta x}=2.$$
$$f'_+(0)=\lim\limits_{\Delta x\to0^+}\df{\ln(1+2\Delta x)-0}{\Delta x}=2.$$
故$f'(0)=2$。

进而,当$x>0$时,$f''(x)=-\df4{(1+2x)^2}$;当$x<0$时,$f''(x)=2$,
又
$$f''_+(0)=\lim\limits_{\Delta x\to0^+}
\df{\frac2{1+2\Delta x}-2}{\Delta x}
=\lim\limits_{\Delta x\to0^+}
\df{-4}{1+2\Delta x}=-4,$$
$$f''_+(0)=\lim\limits_{\Delta x\to0^+}
\df{2\Delta x+2-2}{\Delta x}=2,
$$
故$f''(0)$不存在。综上
$$f''(x)=\left\{\begin{array}{ll}
	-\df4{(1+2x)^2}, & x>0;\\
	2, & x<0.
\end{array}\right.$$
\fin

\bigskip

4.求下列函数的$n$阶导函数
  \begin{enumerate}[(1)]
    \setlength{\itemindent}{1cm}
    \item $y=\sin^2x$;
    \item $y=x\ln x$;
    \item $y=\df{x^2}{1-x}$。
  \end{enumerate}
  
解:(1)$y=\df12(1-\cos2x)$,从而
\begin{align*}
	y'&=\df12\cdot 2\sin2x,\\
	y''&=\df12\cdot 2^2\cos2x=2\sin\left(2x+\df{\pi}2\right),\\
	y'''&=2^2\cos\left(2x+\df{\pi}2\right)=2^2\sin\left(2x+2\cdot\df{\pi}2\right),\\
	\ldots&\ldots\\
	y^{(n)}&=2^{n-1}\sin\left(2x+(n-1)\cdot\df{\pi}2\right)
\end{align*}

(2)
\begin{align*}
	y'&=\ln x+1,\\
	y''&=\df1x,\\
	y'''&=-\df1{x^2},\\
	y^{(4)}&=\df2{x^3},\\
	\ldots&\ldots\\
	y^{(n)}&=(-1)^n\df{(n-2)!}{x^{n-1}}.
\end{align*}

(3)$y=-(x+1)+\df1{1-x}$,
\begin{align*}
	y'&=-1+\df1{(1-x)^2},\\
	y''&=\df2{(1-x)^3},\\
	y'''&=\df{2\cdot 3}{(1-x)^4},\\
	\ldots&\ldots\\
	y^{(n)}&=\df{n!}{(1-x)^{n+1}}.
\end{align*}
\fin

\bigskip

5.已知$f(x)=x^2\ln(1+x)$,求$f^{(n)}(0)$。

解:
% $$f'(x)=2x\ln(1+x)+\df{x^2}{1+x},$$
% 当$x\ne 0$时,
% $$\df{f'(x)}{2x}=\ln(1+x)+\df{x}{2(1+x)},$$
% 两边对$x$求导,可得
% $$\df{f''(x)x-f'(x)}{2x^2}=\df1{1+x}+\df12\df1{(1+x)^2},$$
% 也即
% $$f''(x)x(1+x)^2-f'(x)(1+x)^2=\left(\df32+x\right)x^2.$$
由Leibniz公式,当$n\geq3$时,
\begin{align*}
	f^{(n)}(x)&=x^2[\ln(1+x)]^{(n)}+n2x[\ln(1+x)]^{(n-1)}
	+\df{n(n-1)}22[\ln(1+x)]^{(n-2)}\\
	&=x^2\df{(-1)^{n-1}(n-1)!}{(1+x)^n}+2nx\df{(-1)^{n-2}(n-2)!}{(1+x)^{n-1}}
	+n(n-1)\df{(-1)^{n-3}(n-3)!}{(1+x)^{n-2}}.
\end{align*}
令$x=0$,可得
$$f^{(n)}(0)=(-1)^{n-1}\df{n!}{n-2}.$$
\fin

\begin{center}
	\bf 2.4 隐函数与参数方程求导
\end{center}

\bigskip

1.对下列函数,求$y''(x)$
  \begin{enumerate}[(1)]
    \setlength{\itemindent}{1cm}
    \item $y=\tan(x+y)$;
    \item $y=1+xe^y$。
% 	    \item $x=t(1-\sin t),\;y=t\cos t$。
  \end{enumerate}

解:(1)方程两边对$x$求导,可得
$$y'=\sec^2(x+y)(1+y')=(1+y^2)(1+y'),$$
整理后即为
$$y'=-1-\df1{y^2}.$$
两边再次对$x$求导,可得\ps{根据化简方式的不同,本题结果可能有一些不同的形式}
$$y''=\df{2y'}{y^3}=-\df2{y^3}\left(1+\df1{y^2}\right)
=-2\df{\csc^2(x+y)}{y^3}.$$

(2)方程两边对$x$求导,可得
$$y'=e^y(1+xe^yy'),$$
也即
% $$(1-xe^y)y'=e^y.$$
$$y'=\df{e^y}{1-xe^y}=\df{e^y}{2-y}.$$
% $$xy'=1-\df1{1-xe^y}=1+\df1{2-y}.$$
两边再次对$x$求导,可得
$$y''=e^y\df{y'(2-y)+1}{(2-y)^2}=\df{e^y(e^y+1)}{(y-2)^2}.$$
% $$y'+xy''=-\df{y'}{(y-2)^2}.$$
% $$y'+xy''=-\df{e^y(1+xy')}{(1-xe^y)^2}$$
% $$-e^y(1+xy')y'+(1-xe^y)y''=e^yy'.$$
% $$y''=\df{e^yy'(1-xe^y)+e^ye^y(1+xe^yy')}{(1-xe^y)^2}
% =e^y\df{1+e^{2y}(1+xe^y\frac{e^y}{1-xe^y})}{(1-xe^y)^2}.$$
\fin

\bigskip

2.求曲线$\left\{\begin{array}{l}
x=\df{3t}{1+t^2},\\ y=\df{3t^2}{1+t^2}.
\end{array}\right.$在$(0,0)$和$\left(\df32,\df32\right)$
处的切线方程。

解:
\begin{align*}
	\df{\d x}{\d t}&=\df{3(1+t^2)-3t2t}{(1+t^2)^2}=3\df{1-t^2}{(1+t^2)^2},\\
	\df{\d y}{\d t}&=\df{6t(1+t^2)-3t^22t}{(1+t^2)^2}
	=\df{6t}{(1+t^2)^2},
\end{align*}
故
$$\df{\d y}{\d x}=\df{\df{\d y}{\d t}}{\df{\d x}{\d t}}
=\df{2t}{1-t^2}.$$
在$(0,0)$处$t=0$,$y'_x=0$,切线方程为
$$y=0.$$
在$\left(\df32,\df32\right)$处,$t=1$,此时$y'_x$无意义,但
由$x'_t=0$,$y'_t\ne 0$,故可知切线为铅直方向,切线方程为
$$x=\df32.$$
\fin

\bigskip

3.已知$\left\{\begin{array}{l}
  	x=e^t\cos t\\ y=e^t\sin t
\end{array}\right.$,求$\left.\df{\d y}{\d x}\right|_{t=\frac{\pi}2}$
和$\left.\df{\d^2 y}{\d x^2}\right|_{t=\frac{\pi}2}$。

解:
$$\df{\d x}{\d t}=e^x(\cos t-\sin t),\quad
\df{\d y}{\d t}=e^x(\sin t+\cos t).$$
故
$$\df{\d y}{\d x}=\df{\df{\d y}{\d t}}{\df{\d x}{\d t}}
=\df{\sin t+\cos t}{\cos t-\sin t}.$$
当$t=\df{\pi}2$时,$y'_x=-1$。
进一步
$$\df{\d y'_x}{\d t}=\df{(\cos t-\sin t)(\cos t-\sin t)
-(\sin t+\cos t)(-\sin t-\cos t)}{(\cos t-\sin t)^2}
=\df{2}{(\cos t-\sin t)^2}.$$
故
$$\df{\d^2 y}{\d x^2}
=\df{\df{\d y'_x}{\d t}}{\df{\d x}{\d t}}
=\df{2}{e^x(\cos t-\sin t)^3},$$
从而$y''_{xx}|_{t=\frac{\pi}2}=-2e^{-\frac{\pi}2}$。
\fin

\bigskip

4.设$x(t),y(t)$均三阶可导,试给出$y'''_{xxx}$关于$t$的表达式。

解:
\begin{align*}
	y'_x&=\df{\d y}{\d x}=\df{\df{\d y}{\d t}}{\df{\d x}{\d t}}
	=\df{y'_t}{x'_t};\\
	y''_{xx}&=\df{\d y'_x}{\d x}=\df{\df{\d y'_x}{\d t}}{\df{\d x}{\d t}}
	=\df{\d }{\d t}\left(\df{y'_t}{x'_t}\right)\df{1}{x'_t}
	=\df{y''_{tt}x'_t-y'_tx''_{tt}}{(x'_t)^2}\df{1}{x'_t}
	=\df{y''_{tt}x'_t-y'_tx''_{tt}}{(x'_t)^3},\\
	y'''_{xxx}&=\df{\d y''_{xx}}{\d x}=\df{\df{\d y''_{xx}}{\d t}}{\df{\d x}{\d t}}
	=\df{\d }{\d t}\left(\df{y''_{tt}x'_t-y'_tx''_{tt}}{(x'_t)^3}\right)
	\df{1}{x'_t}\\
	&=\df{(y'''_{ttt}x'_t+y''_{tt}x''_{tt}
	-y''_{tt}x''_{tt}-y'_tx'''_{ttt})(x'_t)^3
	-(y''_{tt}x'_t-y'_tx''_{tt})3(x'_t)^2x''_{tt}}{(x'_t)^6}\df{1}{x'_t}\\
	&=\df{(y'''_{ttt}x'_t-y'_tx'''_{ttt})x'_t
	-3(y''_{tt}x'_t-y'_tx''_{tt})x''_{tt}}{(x'_t)^5}.
\end{align*}
\fin

\bigskip

5.设曲线的极坐标方程为$\rho=\rho(\theta)$,求其对应的直角指标
方程$y=y(x)$相关的导数$y'_x$和$y''_{xx}$关于$\theta$的表达式。

解:$x=\rho(\theta)\cos\theta,\;y=\rho(\theta)\sin\theta$,
以下以$\rho'$和$\rho''$分别表示$\rho'_{\theta}$和$\rho''_{\theta\theta}$。
$$\df{\d x}{\d\theta}=\rho'\cos\theta-\rho\sin\theta,
\quad
\df{\d y}{\d\theta}=\rho'\sin\theta+\rho\cos\theta,$$
故
$$\df{\d y}{\d x}=\df{\df{\d y}{\d\theta}}{\df{\d x}{\d\theta}}
=\df{\rho'\sin\theta+\rho\cos\theta}{\rho'\cos\theta-\rho\sin\theta},$$
又
\begin{align*}
	&(\rho'\sin\theta+\rho\cos\theta)'_{\theta}(\rho'\cos\theta-\rho\sin\theta)
	-(\rho'\sin\theta+\rho\cos\theta)(\rho'\cos\theta-\rho\sin\theta)'_{\theta}\\
	&=(\rho''\sin\theta+\rho'\cos\theta+\rho'\cos\theta-\rho\sin\theta)
	(\rho'\cos\theta-\rho\sin\theta)\\
	&-(\rho'\sin\theta+\rho\cos\theta)
	(\rho''\cos\theta-\rho'\sin\theta-\rho'\sin\theta-\rho\cos\theta)\\
	&=\rho^2+2(\rho')^2-\rho\rho'',
\end{align*}
\begin{align*}
	\df{\d}{\d\theta}\left(\df{\d y}{\d x}\right)
	&=\df{(\rho'\sin\theta+\rho\cos\theta)'_{\theta}(\rho'\cos\theta-\rho\sin\theta)
	-(\rho'\sin\theta+\rho\cos\theta)(\rho'\cos\theta-\rho\sin\theta)'_{\theta}}
	{(\rho'\cos\theta-\rho\sin\theta)^2}\\
	&=\df{\rho^2+2(\rho')^2-\rho\rho''}{(\rho'\cos\theta-\rho\sin\theta)^2},
\end{align*}
故
$$
\df{\d^2y}{\d x^2}
=\df{\df{\d}{\d\theta}\left(\df{\d y}{\d x}\right)}{\df{\d x}{\d\theta}}
=\df{\rho^2+2(\rho')^2-\rho\rho''}{(\rho'\cos\theta-\rho\sin\theta)^3}.
$$
\fin

\begin{center}
	\bf 2.5 相关变化率
\end{center}

\bigskip

1.质点$P$沿抛物线$x=y^2(y>0)$移动。$P$的横坐标$x$的变化速度为$5$cm/s。
当$x=9$cm时,点$P$到原点的距离的变化速率是多少?

解:点$P(x,y)$到原点的距离$d=\sqrt{x^2+y^2}$,则
$$\df{\d d}{\d t}=\df{\sqrt{x^2+y^2}}{\d t}
=\df{xx'_t+yy'_t}{\sqrt{x^2+y^2}}.$$
当$x=9$,$x'_t=5$时,
$$
	y=3,\quad
	y'_t=\df{\d y}{\d x}{\df{\d x}{\d t}}
	=\df{x'_t}{x'_y}=\df5{2y}=\df56.
$$
带入前式,可得所求变化率为$\df{9\cdot 5+3\cdot\df56}{3\sqrt{10}}
=\df{95}{6\sqrt{10}}$。\fin

\bigskip

2.垂直向上发射一枚火箭,在其起飞点100km外设置一个观察站,在观察仰角
为$\pi/4$时,测得仰角的增加率为$0.1$弧度每秒,求此时火车的上升速率。

解:如图,
\begin{center}
	\includegraphics[width=5cm]{./images/ch02/Rocket.jpg}
\end{center}
火箭的高度$H(t)=S\tan\theta(t)$,由此可知
$$H'(t)=S\sec^2\theta(t)\theta'(t),$$
将$S=100$,$\theta(t)=\df{\pi}4$,$\theta'(t)=0.1$代入即得
指定时刻火箭上升的速率为$20$km/s。\fin

\bigskip

3.长度为$6$米的梯子靠在墙角,梯子底部距离墙角$5$米,某一时刻梯子底部开始
向远离墙角的方向滑动,滑动的速度为$0.2$米每秒,问
\begin{enumerate}[(1)]
  \setlength{\itemindent}{1cm}
  \item 此时梯子顶部下滑的速度是多少?
  \item 由梯子、墙面和地面构成的三角形的面积随时间的变化率是多少?
  \item 梯子和地面的夹角的以怎样的速率变化?
\end{enumerate}

解:如图,
\begin{center}
	\includegraphics[width=5cm]{./images/ch02/ladder.jpg}
\end{center}
(1)显然$x^2(t)+y^2(t)=36$,两边求导可得
$$x'(t)x(t)+y'(t)y(t)=0\quad
\Rightarrow\quad y'(t)=-\df{x'(t)x(t)}{y(t)},$$
带入$x(t)=5,x'(t)=0.2,y(t)=\sqrt{11}$,可得指定时刻
梯子顶部的下滑速度为$-\df1{\sqrt{11}}$m/s;

(2)$S(t)=\df12x(t)y(t)$,故
$$S'(t)=\df12[x'(t)y(t)+x(t)y'(t)],$$
带入$x(t)=5,x'(t)=0.2,y(t)=\sqrt{11},y'(t)=-\df1{\sqrt{11}}$,
可得所求三角形面积的变化速率为$-\df{7}{5\sqrt{11}}$m$^2$/s;

(3)$\theta(t)=\arctan\df{y(t)}{x(t)}$,故
$$\theta'(t)=\df1{1+\frac{y^2(t)}{x^2(t)}}
\df{y'(t)x(t)-y(t)x'(t)}{x^2(t)}=\df{y'(t)x(t)-y(t)x'(t)}{x^2(t)+y^2(t)},$$
带入$x(t)=5,x'(t)=0.2,y(t)=\sqrt{11},y'(t)=-\df1{\sqrt{11}}$,
可得指定时刻梯子和地面夹角的变化率为$-\df1{5\sqrt{11}}$弧度/s。\fin

\bigskip

4.半径为$a$的圆球渐渐沉入盛有水的半径为$b(b>a)$的圆柱形容器中,若
球的下降速度恒为$c$,求球浸没入水中恰好一半时,容器内水面上升的速率。
(提示:球冠的体积$V=\df{\pi}3(3R-H)H^2$,其中$H$为球冠的高度)

解:设$t$时刻,没入水中的球冠高度为$h(t)$,则此时没入水中的球冠体积
$$V(t)=\df{\pi}3(3a-h(t))h^2(t),$$
从而没入水中的球冠体积变换率
$$V'(t)=\pi(2ah(t)-h^2(t))h'(t),$$
相应地,桶内水面升高的速率为
$$H'(t)=\df{V'(t)}{\pi [b^2-a^2+(a-h)^2]}
=\df{2ah(t)-h^2(t)}{\pi [b^2-a^2+(a-h)^2]}h'(t),$$
代入$h(t)=a,h'(t)=c$,即得所求时刻水面上升的速率为$\df{a^2c}{b^2-a^2}$。
\fin

\begin{center}
	\bf 2.6 微分
\end{center}

\bigskip

1.设$y=x^3-2x$,
\begin{enumerate}[(1)]
  \setlength{\itemindent}{1cm}
  \item 计算在$x=2$处当$\Delta x$分别为$1,\;0.1,\;
  0.01$时的$\Delta y$和$\d y$;
  \item 写出$x=2$时$y$的微分表达式。
\end{enumerate}

解:
\begin{align*}
	\Delta y&=[(x+\Delta x)^3-2(x+\Delta x)]-(x^3-2x)
	=(3x^2-2)\Delta x+3x\Delta x^2+\Delta x^3,\\
	\d y&=(3x^2-2)\d x=(3x^2-2)\Delta x.
\end{align*}
当$x=2$时,
\begin{align*}
	\Delta y|_{x=2}&
	=10\Delta x+6\Delta x^2+\Delta x^3,\\
	\d y|_{x=2}&=10\Delta x.
\end{align*}
进而
\begin{enumerate}[(i)]
  \item 当$x=2,\Delta x=1$时,$\Delta y=17,\;\d y=10$;
  \item 当$x=2,\Delta x=0.1$时,$\Delta y=1.061,\;\d y=1$;
  \item 当$x=2,\Delta x=0.01$时,$\Delta y=0.100601,\;\d y=0.1$。
\end{enumerate}
\fin

\bigskip

2.已知$2^{xy}=x-y$,求$\d y|_{x=0}$。

解:已知方程两边求微分,可得
$$2^{xy}\ln2(x\d y+y\d x)=\d x-\d y,$$
进而
$$\d y=\df{1-y2^{xy}\ln2}{1+x2^{xy}\ln2}\d x.$$
$x=0$时,$y=-1$,带入即得
$$\d y|_{x=0}=(1+\ln2)\d x.$$
\fin

\bigskip

3.略。

\bigskip

4.设$f(x)=\lim\limits_{t\to\infty}t^2
\left[e^{x+\frac2t}-e^x\right]\sin\df xt$,求$\d f(x)$。

解:
\begin{align*}
	f(x)&=\lim\limits_{t\to\infty}t^2
	\left[e^{x+\frac2t}-e^x\right]\sin\df xt
	=e^x\lim\limits_{t\to\infty}t^2
	\left(e^{\frac2t}-1\right)\df xt\\
	&=e^x\lim\limits_{t\to\infty}t^2\df2t\df xt
	=2xe^x.
\end{align*}
故
$$\d f(x)=(2xe^x)'\d x=2e^x(1+x)\d x.$$
\fin

\bigskip

5.已知当$h\to 0$时,
$$f(x+2h)-f(x)=h\sqrt{x^2+2x}+\circ(h),$$
求$\d\left[f\left(\df1x\right)\right]$。

解:由已知
$$\lim\limits_{h\to0}\df{f(x+2h)-f(x)}{2h}
=\lim\limits_{h\to0}\df{h\sqrt{x^2+2x}+\circ(h)}{2h}
=\df{\sqrt{x^2+2x}}2.
$$
也即$f'(x)=\df{\sqrt{x^2+2x}}2$,故
$$\d\left[f\left(\df1x\right)\right]
=f'\left(\df1x\right)\left(-\df1{x^2}\right)\d x
=-\df{\sqrt{1+2x}}{2x^2|x|}\d x.$$
\fin

\bigskip

6.设$a>0$,$|x|<<a$(表示$|x|$远远小于$a$),证明近似公式
$$\sqrt[n]{a^n+x}\approx a+\df{x}{na^{n-1}}.$$

证:记$f(x)=\sqrt[n]{a^n+x}$,显然$f(0)=a$,
又
$$f'(x)=\df1n(a^n+x)^{\frac1n-1}\quad
\Rightarrow f'(0)=\df1{na^{n-1}}.
$$
故由微分的几何意义,当$|x|<<a$时,总有
$$\sqrt[n]{a^n+x}\approx f(0)+f'(0)x
=a+\df{x}{na^{n-1}}.$$
\fin

\newpage

\begin{center}
	\Large\bf 第二章小结
\end{center}

{\bf 1.导数的概念}

$\bullet$熟练使用导数的各种不同符号,理解其意义:
$$f\,'(x_0),\quad \left.\df{\d y}{\d x}\right|_{x=x_0},\quad\left.\df{\d
}{\d x}y\right|_{x=x_0},\quad \left.\df{\d }{\d x}f(x)\right|_{x=x_0}\quad
y'_x|_{x=x_0},\quad \dot{y}(x_0)$$
$$y'',\quad y''_{xx},\quad \df{\d^2y}{\d x},
\quad \df{\d}{\d x}\left(\df{\d y}{\d x}\right),\quad
\quad \df{\d}{\d x}y',$$
$$f^{\,(n)}(x),\quad \df{\d^n y}{\d x^n},\quad
\quad \df{\d}{\d x}\left(\df{\d^{n-1} y}{\d x^{n-1}}\right),\quad
\quad \df{\d}{\d x}y^{(n-1)}.$$

$\bullet$导数的几何意义:切线(割线的极限位置)的斜率。

注意:可导必存在切线,反之不然。

切线方程:$y=f(x_0)+f'(x_0)(x-x_0)$

法线方程:$y=f(x_0)-\df1{f'(x_0)}(x-x_0)\quad(f'(x_0)\ne 0)$

$\bullet$ $f(x)$在一点可导,则一定在该点连续,反之不然(连续点可能是“尖点”,
例如$y=|x|$在$x=0$处)

例题:已知$f(x)$在$x=0$连续,且$\limx{0}\df{f(x)}x=A$,
证明$f(x)$在$x=0$可导,并求$f'(0)$。

$\bullet$正确区分各种符号,例如:$f'(x_0)$,$[f(x_0)]'$,$f'(x)$,$[f(x)]'$,
$f'_+(x_0)$,$f'(x_0+0)$,\ldots

{\bf 2.导数的计算}

$\bullet$分段定义的函数的导数,根据分段的情况,有时需要使用导数的定义
来计算分段点上的导数。

$\bullet$求导法则:
\begin{enumerate}[(1)]
  \setlength{\itemindent}{1cm}
  \item $[u(x)\pm v(x)]'=u'(x)\pm v'(x)$ 
  \item $[u(x)v(x)]' =u'(x)v(x)+u(x)v'(x)$ 
  \item $\left[\df{u(x)}{v(x)}\right]'
  =\df{u'(x)v(x)-u(x)v'(x)}{v^2(x)}\;\;(\mbox{假设}v(x)\ne 0)$
  \item $[u^{-1}(y)]_y'=\df1{u'_x(x)}$
  \item $[u(v(x))]'=u'(v(x))v'(x)$
\end{enumerate}

难点:正确理解和使用反函数的求导法则!

$\bullet$常用的导函数公式(略)

要求:所求的公式都必须熟练记忆,并且掌握推导的方法!

注意:形如$f(x)^{g(x)}$的函数的求导方法

{\bf 3.高阶导数}

$\bullet$多项式函数的高阶导函数:$n$次多项式函数超过$n$阶的
导函数恒为零。

$\bullet$各种特殊函数计算高阶导函数的技巧:$\ln(1+x)$,$\df1{1+x}$,$\ln\df{1+x}{1-x}$,
$\df{P_n(x)}{x+a}$,$\sin x$,$\cos x$,$x^ne^x$,$x^n\sin x$,
$e^x\sin x$,\ldots

$\bullet$Leibniz公式:
$$\left[u(x)v(x)\right]^{(n)}=
\sum\limits_{k=0}^nC_n^ku^{(n-k)}(x)v^{(k)}(x).$$

{\bf 4.隐函数与参数方程求导}

重点掌握如下的“形式化”推导方法

$\bullet$参数方程的各阶导数
$${\b y'(x)}=\df{\d y}{\d x}=\df{\d y}{\d t}\df{\d t}{\d x}
=\df{\df{\d y}{\d t}}{\df{\d t}{\d x}}={\b \df{y'(t)}{x'(t)}}$$
\begin{align*}
	{\b y''(x)}&=\df{\d^2y}{\d x^2}=\df{\d}{\d x}y'(x)
	=\df{\d}{\d x}\left(\df{y'(t)}{x'(t)}\right)
	=\df{\d}{\d t}\left(\df{y'(t)}{x'(t)}\right)\cdot\df{\d t}{\d x}\\
	&=\df{y''(t)x'(t)-y'(t)x''(t)}{[x'(t)]^2}\cdot\df1{x'(t)}
	={\b \df{y''(t)x'(t)-y'(t)x''(t)}{[x'(t)]^3}}
\end{align*}

$\bullet$反函数的各阶导数:设$f(x)$和$g(y)$互逆,则
\begin{align*}
	g''(y)&=\df{\d^2x}{\d y^2}
	=\df{\d }{\d y}\left(\df{\d x}{\d y}\right)
	=\df{\d }{\d y}\left(\df1{f'(x)}\right)\\
	&=\df{\d }{\b\d x}\left(\df1{f'(x)}\right)\cdot\df{\b\d x}{\d y}
	=-\df{f''(x)}{[f'(x)]^2}\cdot\df1{f'(x)}
	=-\df{f''(x)}{[f'(x)]^3}.
\end{align*}

{\bf 5.相关变化率}

$\bullet$应用题的解题步骤:画图(定义变量)$\to$建立关系$\to$分析求解。

多看一些不同的例题,提高理解题意的能力。

{\bf 6.微分}

$\bullet$可微$\Leftrightarrow$可导。

$\bullet$微分
$$\d y=f'(x)\d x=f'(x)\Delta x,$$
对于自变量$x$,$\d x=\Delta x$

$\bullet$局部线性化和以直代曲
\begin{center}
	\resizebox{!}{6cm}{\includegraphics{./images/ch02/dyDy.pdf}}

	\it $f(x)$在$x_0$可微,意味着在$x_0$附近,$f(x)$可以近似地表示为一个线性函数: 
	$$f(x)\approx y= f(x_0)+f\,'(x_0)(x-x_0)$$
\end{center}

$\bullet$微分的运算法则:可以理解成求导法则的另一种表达形式
\begin{enumerate}[(1)]
  \setlength{\itemindent}{1cm}
  \item $\d (u\pm v)=\d u\pm \d v$
  \item $\d(uv)=v\d u+u\d v$
  \item $\d\df uv=\df{v\d u-u\d v}{v^2}$
  \item $\d g[f(x)]=g'[f(x)]\d f(x)=g'[f(x)]f'(x)\d x$
\end{enumerate}

\fi