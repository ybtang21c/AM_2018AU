% !TEX root = ../0-main-LN.tex

\begin{center}
	\Large\bf
	测验1.1:数列极限

	\kaishu\small (时间60分钟)
\end{center}

1.(10分)用极限的定义证明:$\limn\df1{2^n+n}=0$.\\[3cm]

2.(10分)用极限的定义证明:$\limn(\sqrt{n+1}-\sqrt{n-2})=0$.\\[3cm]

3.(14分)用极限的定义证明:$\limn{(\sqrt{n^2+n}-\sqrt{n^2-n})}=1$.\\[5cm]

4.(14分)已知$\limn a_n=A$,证明$\limn(a_n)^2=A^2$.\\[3cm]

5.(10分)判断数列$\left\{\sin\df{n\pi}2\right\}$是否收敛?给出证明.\\[4cm]

%\hspace{1cm}{\color{white} go}

6.(10分)判断数列$\left\{(-1)^{1+\cos n\pi}\right\}$是否收敛?给出证明.\\[4cm]

判断以下命题的正误(若正确给出证明,若错误举反例)

7.(8分)命题1:数列$\{a_n\}$有界,数列$\{b_n\}$收敛,则$\{a_nb_n\}$收敛.\\[3cm]

8.(8分)命题2:数列$\{a_n\}$无界,数列$\{b_n\}$有界,则$\{a_nb_n\}$无界.\\[3cm]

9.(8分)命题3:数列$\{a_n\}$无界,数列$\{b_n\}$无界,则$\{a_nb_n\}$无界.\\[3cm]

10.(8分)命题4:数列$\{a_n\}$有界,数列$\{b_n\}$收敛,则$\{a_nb_n\}$有界。

\newpage

\begin{center}
	\Large\bf 参考解答
\end{center}

1.用极限的定义证明:$\limn\df1{2^n+n}=0$.

证:对任意$\e>0$,令$N=[-\log_2\e]+1$,则对任意$n>N$,有
$$\left|\df1{2^n+n}-0\right|=\df1{2^n+n}<\df1{2^n}
<\df1{2^N}=\df1{2^{[-\log_2\e]+1}}<\df1{2^{-\log_2\e}}=\e,$$
即证。\fin

\bs
2.用极限的定义证明:$\limn(\sqrt{n+1}-\sqrt{n-2})=0$.

证:对任意$\e>0$,令$N=[9/\e^2]+1$,则对任意$n>N$,有
\begin{align*}
	|\sqrt{n+1}-\sqrt{n-2}|
	&=\df{(\sqrt{n+1}-\sqrt{n-2})(\sqrt{n+1}+\sqrt{n-2})}
	{\sqrt{n+1}+\sqrt{n-2}}
	=\df3{\sqrt{n+1}+\sqrt{n-2}}\\
	&<\df3{\sqrt{n}}<\df3{\sqrt{N}}<\df3{\sqrt{[9/\e^2]+1}}
	<\df3{\sqrt{9/\e^2}}=\e.
\end{align*}
即证。\fin

\bs
3.用极限的定义证明:$\limn{(\sqrt{n^2+n}-\sqrt{n^2-n})}=1$.

证:对任意$\e>0$,令$N=[1/\e]+1$,则对任意$n>N$,有
\begin{align*}
	|\sqrt{n^2+n}&-\sqrt{n^2-n}-1|
	=\df{(\sqrt{n^2+n}-\sqrt{n^2-n}-1)(\sqrt{n^2+n}+\sqrt{n^2-n}+1)}
	{\sqrt{n^2+n}+\sqrt{n^2-n}+1}\\
	&=\df{2n-2\sqrt{n^2-n}-1}{\sqrt{n^2+n}+\sqrt{n^2-n}+1}
	<\df{2(n-\sqrt{n^2-n})}{\sqrt{(n+1)^2}+\sqrt{(n+1)^2}+1}\\
	&=\df{2(n-\sqrt{n^2-n})}{2n+3}
	<\df{n-\sqrt{n^2-n}}{n+1}\\
	&=\df{(n-\sqrt{n^2-n})(n+\sqrt{n^2-n})}{(n+1)(n+\sqrt{n^2-n})}\\
	&=\df{n}{(n+1)(n+\sqrt{n^2-n})}<\df1n<\df1N
	=\df1{[1/\e]+1}<\df1{1/\e}=\e,
\end{align*}
即证。\fin

\bs
4.已知$\limn a_n=A$,证明$\limn(a_n)^2=A^2$.

证:应为$\{a_n\}$收敛,故由极限的有界性,存在$M$,对任意
$n\in\mathbb{Z}_+$,有$|a_n|\leq M$。

对任意$\e>0$,由$\limn a_n=A$,存在$N$,对任意$n>N$,有
$|a_n-A|<\e$,由此可知,对任意$n>N$,均有
$$|(a_n)^2-A^2|=|a_n+A||a_n-A|<(|M|+|A|)\e,$$
其中$|M|+|A|$为常数,故有定义可知$\limn(a_n)^2=A^2$.
\fin

\bs
5.判断数列$\left\{\sin\df{n\pi}2\right\}$是否收敛?给出证明.

答:不收敛。因为当$n=4k+1(k\in\mathbb{Z}_+)$时,总有
$$\sin\df{(4k+1)\pi}2=\sin\left(2k\pi+\frac{\pi}2\right)=1,$$
故原数列的子列$\left\{a_{4k+1}\right\}$收敛于$1$。

又因为当$n=2k(k\in\mathbb{Z}_+)$时,总有
$$\sin\df{2k\pi}2=\sin\left(k\pi\right)=0,$$
故原数列的子列$\left\{a_{2k}\right\}$收敛于$0$。

综上,由子数列的性质可知原数列发散。\fin

\bs
6.判断数列$\left\{(-1)^{1+\cos n\pi}\right\}$是否收敛?给出证明.

答:收敛。因为$\cos n\pi=(-1)^{n+1}$,故该数列即为
$$(-1)^0,(-1)^2,(-1)^0,(-1)^2,(-1)^0,(-1)^2,(-1)^0,(-1)^2,\ldots$$
也即常数数列$\{1\}$,该数列显然收敛。\fin

\bs
判断以下命题的正误(若正确给出证明,若错误举反例)

7.命题1:数列$\{a_n\}$有界,数列$\{b_n\}$收敛,则$\{a_nb_n\}$收敛.

答:错误!反例:$a_n=(-1)^n$,$b_n=1$。\fin

\bs
8.命题2:数列$\{a_n\}$无界,数列$\{b_n\}$有界,则$\{a_nb_n\}$无界.

答:错误!反例:$a_n=n$,$b_n=0$。\fin

\bs
9.命题3:数列$\{a_n\}$无界,数列$\{b_n\}$无界,则$\{a_nb_n\}$无界.

答:错误!反例:$a_n=[1+(-1)^n]n$,$b_n=[1-(-1)^n]n$。\fin

\bs
10.命题4:数列$\{a_n\}$有界,数列$\{b_n\}$收敛,则$\{a_nb_n\}$有界。

答:正确!证明如下:

$\{a_n\}$有界,故存在$M_1$,对任意$n\in\mathbb{Z}_+$,均有
$|a_n|\leq M_1$;

$\{b_n\}$收敛,故由极限的有界性,存在$M_2$,对任意$n\in\mathbb{Z}_+$,均有
$|b_n|\leq M_2$;

综上,对任意$n\in\mathbb{Z}_+$,均有
$$|a_nb_n|\leq M_1M_2,$$
即证。\fin